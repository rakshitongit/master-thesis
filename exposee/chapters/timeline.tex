\chapter{Work Packages and Timeline}
\label{chap:wps}
In this section, we define the work packages and timeline to fulfill the specified tasks in the following months.

\section{Work Packages}
We divide our thesis into Work packages (WP) or Tasks.
We can use this to track the progress of the thesis and get feedback.

\paragraph{WP1 (Introduction):} In this package, we would like to start by introducing the topic, identifying the problems by defining the problem statement, and finally, defining our research and solution approach.
\begin{enumerate}
    \item Motivation
    \item Problem Statement
    \item Research and Solution Approach
\end{enumerate}

\paragraph{WP2 (Foundation):} In this package, we would like to define some terminologies according to the research (e.g., Continous Experimentation, Low-Code, etc.)

\paragraph{WP3 (Related Work):} Here, we would like to find some State-of-the-art research and compare them with the research questions or requirements we define for the thesis.
\begin{enumerate}
    \item State-of-the-art research
    \item Comparison
\end{enumerate}

\paragraph{WP4 (Design Science Research):} Here, we explain our research approach and the DSR in depth.
	\begin{itemize}
		\item DSR Method and Process
		\item Design Principles (DP): We define various DPs from literature reviews on lean development, UI prototyping, etc.
		\item Design Features (DF)
	\end{itemize}

\paragraph{WP5 (Design):} Here, we would like to design our tool as per the LEAN development cycle, containing the Build, Measure, and Learn phases.
\begin{enumerate}
    \item Build: We would like to build the tool for the designers so that they can create prototypes and experiment with the users.
    \item Measure: In this step, we would assign the experiment variants and assign tasks to the users to measure their feedback.
    \item Learn: Finally, we would learn from the experiment by analyzing and improving the prototype. This finishes one cycle iteration, and the designers repeat the steps with the updated prototype.
\end{enumerate}

\paragraph{WP6 (Implementation):} In this package, we would like to implement the prototypes by developing the meta-models, defining the architectures, and coding them for visualization of the prototypes.
\begin{enumerate}
    \item Architecture: We would define the architecture for the experimentation server (containing experimentation and task manager) and the prototypes (containing prototype manager)
    \item Technologies used: Here, we define the technologies that we use for the implementation of the platform 
    \item Demonstration: Developing, Testing, and Deployment of the application (using docker\footnote{Docker: \url{https://www.docker.com/}})
\end{enumerate}

\paragraph{WP7 (Evaluation):} In this package, we would test our prototypes by creating the experiments and the tasks for the users and measuring feedback from them.
\begin{enumerate}
    \item Create a user case study and assign experiments and tasks to the users 
    \item Analyze the measurements
    \item Limitations and Risks
\end{enumerate}

\paragraph{WP8 (Conclusion):} Here, we conclude the thesis and write the scope for future work.
\begin{enumerate}
    \item Conclusion
    \item Future work
\end{enumerate}

\section{GanttChart for Timeline}
% \subsection{GanttChart Simple}
% \begin{ganttchart}
% 	[today=0, %"TODAY" vertical line
% 	]{1}{26}
% 	\gantttitle[title label node/.append style={below left=7pt and -3pt}]{WEEKS:\quad1}{1}
% 	\gantttitlelist{2,...,26}{1} \\
% 	\ganttgroup[progress=0]{WP1}{1}{4} \\
% 	\ganttgroup[progress=0]{WP2}{5}{6} \\
% 	\ganttgroup[progress=0]{WP3}{7}{16} \\
% 	\ganttgroup[progress=0]{WP4}{16}{23} \\
% 	\ganttgroup[progress=0]{WP5}{21}{26}\\
	
% 	\ganttmilestone{Milestone 1}{4}\\
% 	\ganttmilestone{Milestone 2}{8}\\
% 	\ganttmilestone{Milestone 3}{12}\\
% 	\ganttmilestone{Milestone 4}{16}\\
% 	\ganttmilestone{Milestone 5}{20}
% \end{ganttchart}


% \subsection{GanttChart Detailed}
This section gives a rough estimation of the time required for the WPs execution.
We divide the entire time period of the thesis into 26 weeks, with the first four weeks dedicated to the Proposal or the Exposée.

\begin{ganttchart}
	[today=0, %"TODAY" vertical line
	vgrid,
    hgrid,
	progress label text={},
	]{1}{26}
	\gantttitle[title label node/.append style={below left=7pt and -3pt}]{WEEKS:\quad1}{1}
	\gantttitlelist{2,...,26}{1} \\
	\ganttgroup[progress=0]{WP1}{1}{3} \\
	\ganttgroup[progress=0]{WP2}{5}{6} \\
	\ganttgroup[progress=0]{WP3}{2}{4} \\
	\ganttgroup[progress=0]{WP4}{7}{9} \\
		\ganttbar[progress=0]{\textbf{WP 4.1}}{7}{7} \\
		\ganttbar[progress=0]{\textbf{WP 4.2}}{8}{8}\\
		\ganttbar[progress=0]{\textbf{WP 4.3}}{9}{9}\\
	\ganttgroup[progress=0]{WP5}{8}{11}\\
		\ganttbar[progress=0]{\textbf{WP 5.1}}{8}{9}\\
		\ganttbar[progress=0]{\textbf{WP 5.2}}{9}{11}\\
		\ganttbar[progress=0]{\textbf{WP 5.3}}{10}{11}\\
	\ganttgroup[progress=0]{WP6}{11}{20}\\
		\ganttbar[progress=0]{\textbf{WP 6.1}}{11}{16}\\
		\ganttbar[progress=0]{\textbf{WP 6.2}}{14}{15}\\
		\ganttbar[progress=0]{\textbf{WP 6.3}}{13}{20}\\
	\ganttgroup[progress=0]{WP7}{19}{23}\\
	\ganttbar[progress=0]{\textbf{WP 7.1}}{19}{21}\\
	\ganttbar[progress=0]{\textbf{WP 7.2}}{21}{23}\\
	\ganttbar[progress=0]{\textbf{WP 7.3}}{22}{23}\\
	\ganttgroup[progress=0]{WP8}{22}{24}\\
	% \ganttgroup[progress=0]{Buffer}{25}{26}\\
	% \ganttmilestone{Milestone 1}{4}\\
	% \ganttmilestone{Milestone 2}{8}\\
	% \ganttmilestone{Milestone 3}{12}\\
	% \ganttmilestone{Milestone 4}{16}\\
	% \ganttmilestone{Milestone 5}{20}
\end{ganttchart}