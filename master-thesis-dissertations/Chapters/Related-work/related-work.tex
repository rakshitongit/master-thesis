%!TEX root = ../thesis.tex
%*******************************************************************************
%****************************** Related Work Chapter **********************************
%*******************************************************************************
\chapter{Related Work}
\label{chap:relatedWork}
\newcolumntype{M}[1]{>{\centering\arraybackslash}m{#1}}
% **************************** Define Graphics Path **************************
\ifpdf
    \graphicspath{{Chapters/Related-work/Figs/}{Chapters/Related-work/Figs/}{Chapters/Related-work/Figs/}}
\else
    \graphicspath{{Chapters/Related-work/Figs/}{Chapters/Related-work/Figs/}}
\fi
In the previous chapter, we explained the evaluation of our solution approach and the tool we developed.
Consequently, in this chapter, we present a comprehensive overview of the related work in two sections. 
The first section (see section \ref{section:related-word:tools}) will discuss the tools commonly used for UI prototyping and UI split testing. 
We will provide an in-depth analysis of the strengths and limitations of each tool and highlight its unique features and functionalities.
The next section (see section \ref{section:related-word:comparison}) focuses on comparing our tool with the existing ones. 
Similarly, in the sections \ref{section:related-word:sota} and \ref{section:related-word:soacomparison} we will discuss some \ac{soa} technologies and compare them. 
We compare different \ac{dr}s we developed in chapter \ref{chap:design} with the other available tools.

\section{Tools}
\label{section:related-word:tools}
This section will explore some of the existing software or \ac{ra} available in the market.
We have identified some existing tools that are commonly used for UI prototyping and A/B testing.
UX designers widely use these tools to create and test UI prototypes with users. 
Some of these tools are industry-standard and have been around for many years, while others are relatively new.
We will briefly describe each tool's key features and capabilities to provide a comprehensive understanding of the tools we will be comparing.
The comparison will help us identify the gaps and limitations of existing tools and evaluate our tool's uniqueness and innovation. 
Through this section, we aim to provide a comprehensive understanding of the existing tools and technologies and highlight our tool's contributions to the field of UI prototyping and UI split testing.

\paragraph{\ac{ra}1 Figma:} 
Figma\footnote{Website for Figma: \url{https://help.figma.com/hc/en-us/articles/360040314193-Guide-to-prototyping-in-Figma}} is a popular \ac{ui} design tool designers and design teams use for prototyping, UI design, and collaborative work. 
It has gained popularity due to its versatile features, user-friendly interface, and real-time collaboration capabilities, making it a go-to tool for many designers.
Figma has a broad range of features that allow designers to create complex UI designs easily. 
It includes an extensive library of UI elements, customizable templates, and vector networks to make designing more manageable. 
Additionally, it offers prototyping features that allow designers to simulate complex user interactions and animations, which can help designers validate their decisions quickly. 
Figma is an excellent tool for teams working on the same design project, as it offers real-time collaboration. 
Designers can work together on the same design file, share feedback, and update the design in real-time, making it an efficient and collaborative tool for design teams.

\paragraph{\ac{ra}2 InVision:}
InVision\footnote{Website for InVision: \url{https://www.invisionapp.com/defined/prototype}} is a UI prototyping and collaboration platform that allows designers to create interactive and animated prototypes for web and mobile applications. 
It offers a range of features, including vector-based design tools, advanced animations and interactions, and an extensive library of pre-built UI components.
Its design tools are based on vector graphics, allowing for easy scaling and resizing of elements.
The platform also offers various commenting and annotation tools, making it easy for team members to communicate and collaborate on design decisions.
InVision also offers advanced animation and interaction capabilities, allowing designers to create complex, engaging interactions that simulate real-world user experiences. This feature includes support for animations, transitions, and gestures and the ability to create interactive elements such as buttons, menus, and forms.

\paragraph{\ac{ra}3 Axure:}
Axure\footnote{Website for InVision: \url{https://www.axure.com/prototype}} is another popular prototyping tool used in the industry. 
It is a wireframing and prototyping tool that allows designers to create complex interactions and dynamic content. 
With Axure, designers can create interactive prototypes with conditional logic, animations, and data-driven interactions. 
It also offers collaboration features, which enable teams to work together on the same project in real-time.
Its ability to handle complex logic and conditional interactions sets it apart from many other prototyping tools. 
However, the learning curve can be steep, and there may be better choices for designers looking for a quick and easy way to create simple prototypes.
Finally, Axure offers integrations with other design and collaboration tools, which can benefit teams using multiple tools in their workflow.

\paragraph{\ac{ra}4 Adobe XD:}
Adobe XD\footnote{Website for Adobe XD: \url{https://www.adobe.com/products/xd/learn/get-started-xd-prototype.html}} is a popular tool for designing and prototyping digital products, including websites, mobile apps, and other interactive experiences. 
It is widely used by UX designers, product managers, and other professionals in the design industry. 
Adobe XD's capability to produce interactive prototypes is one of its primary characteristics.
With XD, designers can create clickable, interactive mockups of their designs, allowing them to test user flows and interactions before writing any code. 
These prototypes can also be shared with stakeholders and users for feedback.
In addition to prototyping, it is possible to integrate external plugins to conduct A/B testing in Adobe XD. 
One such plugin is UserTesting, which allows designers to recruit participants and set up tasks for A/B testing directly within Adobe XD. 

\paragraph{\ac{ra}5 Proto.io:} 
Proto.io\footnote{Website for Adobe XD: \url{https://proto.io/developers/}} is a powerful UI prototyping tool widely used in the industry. 
It offers a variety of features to create interactive prototypes, including the ability to add animations, transitions, and gestures. 
Proto.io provides an intuitive drag-and-drop interface and supports multiple platforms, including iOS, Android, and the web.
One of the standout features of Proto.io is its ability to simulate the final product. 
This feature allows designers to create prototypes that look and feel like the final product, providing a more realistic user experience. 
Proto.io also provides advanced collaboration features, making sharing and collaborating on designs with stakeholders and team members easy. 
Another key feature of Proto.io is its ability to create A/B tests. The tool offers a dedicated A/B testing feature, which allows designers to create multiple versions of a design and test them against each other. 
The tool also provides detailed analytics and user feedback, helping designers make informed design decisions.

\paragraph{\ac{ra}6 Google Optimize:}
Google Optimize\footnote{Website for Google Optimize: \url{https://developers.google.com/optimize/devguides/experiments?technology=ga4}} is an A/B testing and personalization tool developed by Google. 
It is a cloud-based tool that can help businesses optimize their website or app by creating and running experiments. 
Google Optimize has a user-friendly interface allowing users to create experiments without coding knowledge. 
One of the key features of Google Optimize is the ability to create A/B tests with different variations of a website or app. 
Users can set up experiments to test various elements of their website or app, such as headlines, images, or calls to action. 
Google Optimize also allows users to create multivariate tests that test multiple elements simultaneously.
Another important feature of Google Optimize is the ability to create personalization experiments. 
Users can create targeted experiences for specific audiences based on location, behavior, or demographics. 
This allows businesses to create more relevant and engaging experiences for their users.
Google Optimize also integrates with other Google products, such as Google Analytics, allowing users to analyze experiment results and gain insights into user behavior. 
Additionally, Google Optimize supports third-party integrations.

\paragraph{\ac{ra}7 VWO:}
VWO\footnote{Website for VWO: \url{https://help.vwo.com/hc/en-us/articles/360021171954-How-to-Create-an-A-B-Test-in-VWO-}} (Visual Website Optimizer) is another popular tool for A/B testing and conversion rate optimization. 
It is a cloud-based platform that enables users to run experiments on their websites or mobile apps to improve the user experience and increase conversions.
VWO offers a variety of features for A/B testing, including split URL testing, heatmaps, session recordings, surveys, and personalization. 
Users can create and run experiments with a simple WYSIWYG\footnote{Website for WYSIWYG: \url{https://www.howtogeek.com/752396/what-is-a-wysiwyg-editor/}} editor or use custom code for more advanced experiments.
VWO also provides a powerful targeting engine that allows users to target specific segments of their audience, such as first-time visitors, returning visitors, or users who have abandoned a shopping cart. 
This targeting engine helps to ensure that experiments are relevant to the user and are more likely to lead to a positive outcome.

\paragraph{\ac{ra}8 Convertize:}
Convertize\footnote{Website for Convertize: \url{https://docs.convertize.io/docs/how-to-do-ab-testing/}} is a website optimization tool that provides a wide range of features for A/B testing, personalization, and targeting. 
It enables marketers to create and test multiple websites or landing page versions to determine which design, layout, or content performs best for their target audience.
Convertize offers a visual editor allowing users to create different variations of their website without coding skills. 
Users can drag and drop elements, such as images, text, and buttons, to create different versions of their websites, which can then be tested against each other.
One of the unique features of Convertize is its AI-powered Autopilot mode, which allows users to automate the optimization process. With Autopilot, Convertize automatically adapts the website to each visitor by personalizing the content, design, and layout to maximize conversions.
Another notable feature of Convertize is its SmartEditor, which uses a library of pre-built templates and machine learning algorithms to recommend changes to website elements, such as headlines, images, and CTAs, that will likely improve conversions.

\paragraph{\ac{ra}9 Freshmarketer:}
Freshmarketer\footnote{Website for Freshmarketer: \url{https://support.freshmarketer.com/en/support/solutions/folders/50000000186/page/3}} is a comprehensive conversion optimization suite that offers a range of tools to help businesses optimize their website and improve their online performance. 
It allows users to create and execute A/B tests, track visitor behavior, and gather customer feedback from a single platform. 
Freshmarketer's A/B testing tool enables users to create multiple webpage variants and test them against each other to determine which is most effective at achieving a specific goal, such as increasing conversions or decreasing bounce rates. 
The platform also offers advanced targeting options to help users deliver personalized experiences to particular audience segments.
In addition to A/B testing, Freshmarketer offers a range of other optimization tools, including heatmaps, session replays, and form analytics, to help users gain deeper insights into visitor behavior and identify areas for improvement. 
Freshmarketer's customer feedback tools, such as surveys and polls, allow users to collect valuable customer insights and use that feedback to inform future optimization efforts. 

\paragraph{\ac{ra}10 Zoho PageSense:}
Zoho PageSense\footnote{Website for Zoho Pagesense: \url{https://help.zoho.com/portal/en/kb/pagesense/run-a-b-and-split-url-tests/create-and-launch-a-test/articles/a-b-test}} is a web optimization tool designed to help businesses optimize their website's user experience, increase visitor engagement, and boost conversions. 
Zoho PageSense offers many features to help companies to optimize their website, including A/B testing, heatmaps, funnel analysis, and more.
With Zoho PageSense's A/B testing feature, businesses can create multiple variations of their website pages and test them against each other to determine which performs best. 
Zoho PageSense offers a visual editor that allows users to create and edit website page variations without requiring coding knowledge.
In addition to A/B testing, Zoho PageSense offers heatmaps, allowing businesses to visualize how users interact with their website pages. 
This feature enables companies to identify areas of their website that are not performing well and make data-driven decisions to improve the user experience.

\clearpage
\section{Comparison of Tools}
\label{section:related-word:comparison}
In this section, we compare different tools based on their ability to fulfill the DRs we have defined for our project. 
In the table (see \ref{table:related:work:comparision}), we categorize the fulfillment of DRs into three categories: no fulfillment, partial fulfillment, and complete fulfillment. 
It is important to note that these categories are based on our perspective and are not mentioned anywhere else. 
When we say a tool has no fulfillment, we mean it does not satisfy the DR. On the other hand, partial fulfillment means that the tool only partially meets the DR. 
In contrast, complete fulfillment indicates that the tool fully satisfies the DR. 
This categorization helps us compare and evaluate the tools based on our needs and requirements.

In comparing the different tools based on the \ac{dr}s (see table \ref{table:related:work:comparision}), we can see that only some tools fulfilled all the \ac{dr}s. 
\ac{ra}5 and \ac{ra}6 fulfilled the most \ac{dr}s, with \ac{ra}5 fulfilling five out of ten \ac{dr}s and \ac{ra}6 fulfilling six out of ten DRs.
For \textit{Heterogeneous Users (DR1)}, only \ac{ra}5, \ac{ra}6, \ac{ra}9, and \ac{ra}10 fulfilled this requirement completely. Most of the other tools only partially fulfilled this requirement.
Most tools fulfilled this requirement for \textit{Iterative Design (DR2)}, but some only partially fulfilled it.
For \textit{Easy Development (DR3)}, \ac{ra}1, \ac{ra}2, \ac{ra}3, \ac{ra}4, and \ac{ra}5 fulfilled this requirement. 
The rest of the tools did not fulfill it.
For \textit{Integrate Data Models (DR4)}, only \ac{ra}3 fulfilled this requirement completely. 
Most of the other tools only partially fulfilled this requirement.
For \textit{Classified UI Variants (DR5)} and \textit{Conduct Split Tests (DR6)}, \ac{ra}5, \ac{ra}6, \ac{ra}7, \ac{ra}8, \ac{ra}9, and \ac{ra}10 fulfilled this requirement, but the rest of the tools still need to.

\begin{table}[htbp!]
  \centering
  \begin{tabular}{| M{2.5em} || M{2em} | M{2em} | M{2em} | M{2.1em} | M{2.1em} | M{2.1em} | M{2em} | M{2em} | M{2em} | M{2.3em} |}
  \hline 
  \multicolumn{11}{|c|}{\textbf{Comparison between different Tools}} \\ 
  \hline
  \textbf{Tool} & \textbf{DR1} & \textbf{DR2} & \textbf{DR3} & \textbf{DR4} & \textbf{DR5} & \textbf{DR6} & \textbf{DR7} & \textbf{DR8} & \textbf{DR9} & \textbf{DR10} \\
  \hline
  \ac{ra}1 & \priority{50} & \priority{100} & \priority{100} & \priority{50} & \priority{0} & \priority{0} & \priority{0} & \priority{50} & \priority{0} & \priority{50} \\
  \hline
  \ac{ra}2 & \priority{50} & \priority{100} & \priority{100} & \priority{50} & \priority{50} & \priority{50} & \priority{0} & \priority{50} & \priority{0} & \priority{50} \\
  \hline
  \ac{ra}3 & \priority{50} & \priority{100} & \priority{100} & \priority{100} & \priority{0} & \priority{0} & \priority{0} & \priority{50} & \priority{0} & \priority{50} \\
  \hline
  \ac{ra}4 & \priority{50} & \priority{100} & \priority{100} & \priority{50} & \priority{50} & \priority{50} & \priority{0} & \priority{50} & \priority{50} & \priority{50} \\
  \hline
  \ac{ra}5 & \priority{100} & \priority{50} & \priority{100} & \priority{0} & \priority{100} & \priority{100} & \priority{50} & \priority{100} & \priority{50} & \priority{100} \\
  \hline
  \ac{ra}6 & \priority{100} & \priority{100} & \priority{0} & \priority{0} & \priority{100} & \priority{100} & \priority{50} & \priority{100} & \priority{100} & \priority{100} \\
  \hline                                   
  \ac{ra}7 & \priority{50} & \priority{100} & \priority{0} & \priority{0} & \priority{100} & \priority{100} & \priority{50} & \priority{100} & \priority{50} & \priority{100} \\
  \hline
  \ac{ra}8 & \priority{50} & \priority{50} & \priority{0} & \priority{0} & \priority{100} & \priority{100} & \priority{50} & \priority{100} & \priority{50} & \priority{50} \\
  \hline
  \ac{ra}9 & \priority{100} & \priority{100} & \priority{0} & \priority{0} & \priority{100} & \priority{100} & \priority{100} & \priority{100} & \priority{50} & \priority{50} \\
  \hline
  \ac{ra}10 & \priority{100} & \priority{100} & \priority{0} & \priority{0} & \priority{100} & \priority{100} & \priority{100} & \priority{100} & \priority{50} & \priority{100} \\
  \hline
  \hline
  \multicolumn{1}{|c}{Legend:} & \multicolumn{3}{c}{No Fulfillment (\priority{0})} & \multicolumn{3}{c}{Partial Fulfilment (\priority{50})} & \multicolumn{4}{c|}{Complete Fulfilment (\priority{100})} \\
  \hline
  \end{tabular}
  \caption[Comparison Between Different Approaches]{Table Comparing Different \ac{ra}s Against \ac{dr}s}
  \label{table:related:work:comparision}
\end{table}

For \textit{Conduct User Tasks (DR7)}, only \ac{ra}9 and \ac{ra}10 fulfilled this requirement completely. The rest of the tools only partially fulfilled this requirement or did not fulfill it.
Similarly, for \textit{Collect User Feedback (DR8)}, most tools partially fulfilled this requirement, except for \ac{ra}5, \ac{ra}6, \ac{ra}7, \ac{ra}9, and \ac{ra}10 which fulfilled this requirement completely.
For \textit{Aggregated Feedback (DR9)}, most tools only partially or still need to fulfill this requirement. 
Only \ac{ra}5 fulfilled this requirement completely.
For \textit{Improvement (DR10)}, most of the tools only partially fulfilled this requirement or still need to fulfill it. 
Only \ac{ra}5, \ac{ra}6, \ac{ra}7, and \ac{ra}10 fulfilled this requirement completely.
Overall, each tool had its strengths and weaknesses in fulfilling the different DRs, and no tool could completely fulfill all our DRs.

\clearpage

\section{State-of-the-Art Technologies}
\label{section:related-word:sota}
This section explores some \ac{soa} or cutting-edge technologies for UI prototyping and A/B testing of UI. 
UX designers widely use these technologies to create and test UI prototypes with users. 
This section aims to provide a comprehensive understanding of the existing \ac{soa} and highlight our technology's contributions to UI prototyping and UI split testing using the \ac{dr}s. 
We will provide a comprehensive overview of each technology's key features and capabilities to understand the these technologies we will compare comprehensively. 
The comparison will help us identify the gaps and limitations of existing technologies and evaluate our technology's uniqueness and innovation.

\paragraph{RA11 Rapid software prototyping approach:}
L. Luqi et al. \cite{paper:prototyping:luqi} propose a rapid software prototyping methodology that utilizes a visual design tool and object-oriented technology. 
The authors argue that traditional software development methodologies need to provide adequate support for rapidly creating prototypes that can be used to communicate design ideas with stakeholders. 
To address this issue, they propose a framework that combines a visual design tool for creating user interfaces with object-oriented technology to build functional prototypes quickly. 
The authors demonstrate the effectiveness of their methodology through a case study, which shows that the proposed framework can significantly reduce the time and effort required to develop functional prototypes.

\paragraph{RA12 Continuous Prototyping approach:}
Lukas Alperowitz et al. \cite{misc:prototyping:lukas} propose a continuous prototyping approach to bridge the gap between design and development in continuous software engineering. 
The authors argue that traditional software development methodologies must provide adequate support for continuous prototyping, a critical aspect of iterative design and development processes. 
To address this issue, they propose a framework that enables designers and developers to continuously prototype and refine their design ideas as part of the software development process. 
The authors demonstrate the effectiveness of their approach through a case study, which shows that continuous prototyping can significantly improve the quality and speed of software development and increase stakeholder satisfaction.

\paragraph{RA13 Data-Driven Approaches to User Interface Design:}
Pimenov et al. \cite{misc:data-driven:pimenov} present a case study on data-driven approaches to user interface design. 
The authors argue that traditional user interface design methods rely on expert opinions and intuition, which may not always reflect the actual needs and preferences of users. 
To address this issue, they propose a data-driven approach that utilizes user feedback and analytics to inform design decisions. 
The authors demonstrate the effectiveness of their approach through a case study, that shows that data-driven design can lead to significant improvements in user satisfaction and task completion rates. 
The paper provides insights into the potential of data-driven approaches to user interface design and highlights the importance of incorporating user feedback and analytics in the design process.

\paragraph{RA14 A Tool for Online Experiment-Driven Adaptation:}
Ilias Gerostathopoulos et al. \cite{misc:experiment:ilias} propose a tool for online experiment-driven adaptation. 
The authors argue that traditional software development approaches must provide adequate support for online experimentation and transformation, which are critical for improving the UX and achieving business goals. 
To address this issue, they propose a framework that enables developers and designers to create and deploy online experiments that can be used to test and optimize different aspects of the software system. 
The authors demonstrate the effectiveness of their approach through a case study, which shows that the proposed tool can significantly improve the effectiveness of online experimentation and adaptation. 
The paper provides insights into the potential of experiment-driven transformation to improve software systems and highlights the importance of incorporating data-driven approaches in software development.

\paragraph{RA15 A promising tool for customer value evaluation:}
Peitsa Hynninen et al. \cite{misc:abtest:marjo} propose using A/B testing as a promising tool for evaluating customer value. 
The authors argue that traditional methods of customer value evaluation, such as surveys and focus groups, may not provide accurate or reliable results due to biases and limitations. 
To address this issue, they propose the use of A/B testing, which is a randomized experiment that compares the effectiveness of two or more alternatives in achieving a specific goal. 
The authors demonstrate the effectiveness of A/B testing through a case study, which shows that A/B testing can provide valuable insights into customer preferences and behavior. 
The paper provides insights into the potential of A/B testing as a tool for customer value evaluation. 
It highlights the importance of incorporating data-driven approaches in marketing and customer experience strategies. 

\clearpage
\section{Comparison of SOAT}
\label{section:related-word:soacomparison}
In our chapter on design chapter \ref{chap:design}, we identified 10 DRs for the development of effective user interfaces. 
To evaluate the \ac{soa} based on these \ac{dr}s, we analyzed five studies, namely \ac{ra}11, \ac{ra}12, \ac{ra}13, \ac{ra}14, and \ac{ra}15 (see table \ref{table:related:work:soacomparision}).
\begin{table}[htbp!]
  \centering
  \begin{tabular}{| M{3em} || M{2em} | M{2em} | M{2em} | M{2.1em} | M{2.1em} | M{2.1em} | M{2em} | M{2em} | M{2em} | M{2.2em} |}
  \hline 
  \multicolumn{11}{|c|}{\textbf{Comparison between different \ac{ra}}} \\ 
  \hline
  \textbf{Tool} & \textbf{DR1} & \textbf{DR2} & \textbf{DR3} & \textbf{DR4} & \textbf{DR5} & \textbf{DR6} & \textbf{DR7} & \textbf{DR8} & \textbf{DR9} & \textbf{DR10} \\
  \hline
  \ac{ra}11 & \priority{0} & \priority{100} & \priority{100} & \priority{50} & \priority{0} & \priority{0} & \priority{0} & \priority{0} & \priority{0} & \priority{100} \\
  \hline
  \ac{ra}12 & \priority{0} & \priority{100} & \priority{100} & \priority{100} & \priority{50} & \priority{0} & \priority{0} & \priority{50} & \priority{0} & \priority{100} \\
  \hline
  \ac{ra}13 & \priority{50} & \priority{50} & \priority{50} & \priority{100} & \priority{0} & \priority{0} & \priority{50} & \priority{100} & \priority{100} & \priority{50} \\
  \hline
  \ac{ra}14 & \priority{100} & \priority{50} & \priority{0} & \priority{0} & \priority{50} & \priority{100} & \priority{50} & \priority{100} & \priority{50} & \priority{50} \\
  \hline
  \ac{ra}15 & \priority{100} & \priority{50} & \priority{0} & \priority{0} & \priority{100} & \priority{100} & \priority{50} & \priority{50} & \priority{50} & \priority{50} \\
  \hline
  \hline
  \multicolumn{1}{|c}{Legend:} & \multicolumn{3}{c}{No Fulfillment (\priority{0})} & \multicolumn{3}{c}{Partial Fulfilment (\priority{50})} & \multicolumn{4}{c|}{Complete Fulfilment (\priority{100})} \\
  \hline
  \end{tabular}
  \caption[Comparison Between Different Software Approaches]{Table Comparing Different \ac{soa}s Against \ac{dr}s}
  \label{table:related:work:soacomparision}
\end{table}

\ac{ra}11 fulfilled DR2, DR3, and DR10, partially fulfilled DR4, and failed to fulfill DR1, DR5, DR6, DR7, DR8, and DR9. 
\ac{ra}12 fulfilled DR2, DR3, DR4, and DR10, partially fulfilled DR5 and DR8, and failed to fulfill DR1, DR6, DR7, DR9. 
\ac{ra}13 partially fulfilled DR1, DR2, DR3, DR7, DR8, and DR10, fulfilled DR4 and DR8, and failed to fulfill DR5 and DR6. 
\ac{ra}14 fulfilled DR1, DR6, and DR8, partially fulfilled DR2, DR5, DR7, and DR9, and failed to fulfill DR3 and DR4. 
Finally, \ac{ra}15 fulfilled DR1, DR5, DR6, and DR10, partially fulfilled DR2, DR7, DR8, and DR9, and failed to fulfill DR3 and DR4.

Overall, we observed that none of the \ac{ra}s fully fulfilled our identified DRs. 
However, \ac{ra}15 achieved the most DRs, with four fulfilled and four partially fulfilled, while \ac{ra}11 and \ac{ra}12 each achieved three fulfilled DRs. 
Our analysis demonstrates the need for continued research in developing effective user interfaces that fulfill all identified DRs.