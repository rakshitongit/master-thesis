%!TEX root = ../thesis.tex
%*******************************************************************************
%****************************** Design Chapter *********************************
%*******************************************************************************

\chapter{Design}

\ifpdf
    \graphicspath{{Chapters/Design/Figs/}{Chapters/Design/Figs/}{Chapters/Design/Figs/}}
\else
    \graphicspath{{Chapters/Design/Figs/}{Chapters/Design/Figs/}}
\fi
%********************************** % Solution Design **************************************
\section{Solution Design}

%********************************** % Design Requirements **************************************
\subsection{Design Requirements}
\label{design:section:designReqs}
Prescriptive knowledge about the design of artifacts, such as software techniques, models, or concepts, is what design science research (DSR) aims can provide. 
Due to this design knowledge, future projects can design artifacts methodically and scientifically with the aid of study and practice. 
As a result, this design and application may provide design-focused information that adds to the DSR knowledge corpus \cite{misc:dsr:henver}.
Every element within a DSR project is built upon and systematically analyzed to add to the overall DSR knowledge corpus.
Several scientific methodologies are employed in the DSR to ensure that this information is presented practically using design requirements (DRs).
There are various ways to retrieve the DRs as explained below.

\paragraph*{Systematic literature reviews:} 
Systematic literature reviews establish a solid basis for knowledge creation. 
They help create theories and identify areas that require more research.
By identifying existing literature and highlighting areas to explore, they also assist in the understanding of the problem space in DSR.
Therefore, a comprehensive DSR project must include systematic literature reviews \cite{misc:dsr:webster}.

\paragraph*{Interviews:}
Interviews are regarded as empirical research to obtain primarily qualitative insights from individuals, such as specialists in a particular subject. 
Suppose the interviewers are a part of the problem's stakeholder group.
In that case, they may be utilized to establish requirements (or meta requirements) for the solution space that have a solid foundation for the problem space.
For example, artifacts can be designed based on design principles (DPs) generated from interviews.
Furthermore, the interviews can also be used to improve and evaluate the designed artifacts \cite{misc:dsr:mayring}. 

\paragraph*{Other Methods:}
Design requirements can also be generated using other methods or combinations of methods.
Other methods include simulations, experiments, case studies, ethnography, and the grounded theory approach \cite{misc:dsr:nickerson, misc:dsr:varshney}. \\\\
For our approach, we are using the systematic literature review technique to generate the Design Requirements (DRs).
\begin{itemize}
  \item \textbf{DR1} \textit{Heterogeneous Users:} While using the prototyping, we should use different users for better research results in the research \cite{article:prototyping:weichbroth,misc:lean:steve, misc:lean:burmeister}. This is because different users may have different needs, preferences, and levels of technical expertise. By including a diverse group of users, you can get a broader range of feedback and insights into how the software performs for different users.
  \item \textbf{DR2} \textit{Different stakeholders for designing and developing software:} Involve project stakeholders by providing them with the necessary communication tools for exchanging and codifying knowledge \cite{article:prototyping:weichbroth, misc:prorotypes:lauff}. Stakeholders will have their perspectives and requirements, and their input can help shape the product's design.
  \item \textbf{DR3} \textit{Diversity in the GUI:} Do not use the commonly used GUIs for testing. This provides diversity while creating the prototypes and gathers requirements from prospective users \cite{article:prototyping:weichbroth}. CE should be added
  \item \textbf{DR4} \textit{Diversity in testing:} Testing of the GUI must be automated, and use functional and usability tests. Functional testing helps ensure the GUI works according to specification and usability testing helps ensure users can use the GUI effectively. And this helps in the identification and preliminary validation of user requirements in the early stages of development \cite{article:prototyping:weichbroth}.
  \item \textbf{DR5} \textit{Governance Mechanisms:} The solution must incorporate governance elements into the process so the platform owner can take the appropriate action against developers and users who abuse the validation process. Therefore, good governance will make sure that the users stay with the platform for a longer period \cite{misc:prototypes:evans,misc:dps:parker}. 
  \item \textbf{DR6} \textit{Different UI Properties:} The UI should be clear, concise, familiar, responsive, consistent, attractive, efficient and forgiving \cite{article:prototyping:weichbroth}. If the UI has the following properties, then it gives a better User Experience (UX).
  \item \textbf{DR7} \textit{Provide users with tasks:} Having users perform specific tasks and observing how they interact with the application helps improve the usability of the software applications. So, we can use task-based usability testing for achieving this. 
  \item \textbf{DR8} \textit{Design Models for software:} The model should be complete, consistent, and correct. Thus, modeling can help developers better understand the project's requirements, identify potential problems and design solutions, and communicate their ideas to other development team members. This helps in creating model-based prototypes with many benefits.
  \item \textbf{DR9} \textit{Diversity in feedback mechanism:} Including qualitative and quantitative feedback when evaluating a product or service is generally a good idea. Qualitative feedback is based on customers' or users' opinions and perceptions, providing insight into their experiences and feelings about the product. This type of feedback can help identify areas where the product is excelling and areas where it can be improved. Quantitative feedback, on the other hand, is based on numerical data and measurements, and it provides a more objective view of the product's performance. This type of feedback can help track trends and make comparisons over time.
  \item \textbf{DR10} \textit{Software product should be easily built:} Using Low/No-code, we can use the visual interfaces and pre-built modules to create software without writing code. This can help users build simple software applications without programming knowledge. This approach can make it easier for non-technical people to build software and automate tasks.
\end{itemize}

%********************************** % Design Principles **************************************
\subsection{Design Principles}

%********************************** % Solution Concept **************************************
\section{Solution Concept}

%********************************** % Build **************************************
\subsection{Build}

%********************************** % Measure **************************************
\subsection{Measure}

%********************************** % Learn **************************************
\subsection{Learn}
