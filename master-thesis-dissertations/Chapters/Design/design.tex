%!TEX root = ../../thesis.tex
%*******************************************************************************
%****************************** Design Chapter *********************************
%*******************************************************************************

\chapter{Solution Design}
\label{chap:design}

\ifpdf
    \graphicspath{{Chapters/Design/Figs/}{Chapters/Design/Figs/}{Chapters/Design/Figs/}}
\else
    \graphicspath{{Chapters/Design/Figs/}{Chapters/Design/Figs/}}
\fi
In the previous chapter, we defined various terminologies and concepts that we are using for our thesis.
Consequently, this chapter presents the solution design for our systematic user feedback evaluation approach as per \ac{dsr}.
Firstly, we define the \ac{dr}s (see section \ref{design:section:designReqs}) to guide the development of our solution.
Secondly, we concretize the \ac{dr}s to derive a set of \ac{dp}s (see section \ref{design:section:designprinciple}) for developing our prototype tool, ensuring that our solution is efficient, user-friendly, and effective in evaluating user feedback.
Lastly, we present the overall solution design (see section \ref{design:section:solutiondesign}) that tracks the \ac{dp}s following the LEAN development cycle. 
The solution design comprises a set of phases, each with its corresponding \ac{dp}s and goals, to ensure that the development of our prototype tool is systematic and efficient.

% Therefore, we explore the \ac{dr}s (see section \ref{design:section:designReqs}), \ac{dp}s (see section \ref{design:section:designprinciple}), and an overall solution design (see section \ref{design:section:solutiondesign} which gives an incite of the creation of our solution approach) to guide the result of our software tool.
% Design is a critical aspect of \ac{se}, providing the blueprint for implementing functional and non-functional requirements. 
% The design phase is crucial in ensuring that software systems are developed efficiently, effectively, and with high quality. 
% We develop our design approach based on the principles of \ac{dsr}. 
% Through this exploration, we will provide an in-depth understanding of the design process, highlighting the key factors that must be considered to create effective software solutions.

%********************************** % Design Requirements **************************************
\section{Design Requirements}
\label{design:section:designReqs}
\ac{dr}s in \ac{dsr} are typically defined as a set of constraints and specifications that must be met by the design artifact to be considered a successful solution. 
These are functional requirements, such as the features or capabilities the design/tool should have.
In our findings of the \ac{dr}s, we focus more on the functional requirements and ignore the non-functional.
To derive a solution, we define the \ac{dr}s with the help of the literature review and a comparison of some tools.
For our research, we conducted a non-systematic literature review by reading research papers and looking at some renowned UI prototyping tools. 
The related literature is available in Appendix \ref{appendix:two:definations}.
In this context, each \ac{dr} refers to a generalized requirement that can be standardized and applied to future software applications.
We covered many topics including \textit{\ac{ui} Prototyping}, \textit{Low-Code/No-Code development}, \textit{Model-Based Software Engineering}, \textit{Continuous Experimentation}, \textit{Task-Based Usability Testing}, and the \textit{LEAN development process}.
The following section presents \textit{nine \ac{dr}s} for our approach \textit{(solution approach)}.
% \paragraph*{Systematic literature reviews:} 
% Systematic literature reviews establish a solid basis for knowledge creation. 
% They help create theories and identify areas that require more research.
% By identifying existing literature and highlighting areas to explore, they also assist in the understanding of the problem space in DSR.
% Therefore, a comprehensive DSR project must include systematic literature reviews \cite{misc:dsr:webster}.
% \paragraph*{Interviews:}
% Interviews are regarded as empirical research to obtain primarily qualitative insights from individuals, such as specialists in a particular subject. 
% Suppose the interviewers are a part of the problem's stakeholder group.
% In that case, they may be utilized to establish requirements (or meta requirements) for the solution space that have a solid foundation for the problem space.
% For example, artifacts can be designed based on design principles (DPs) generated from interviews.
% Furthermore, the interviews can also be used to improve and evaluate the designed artifacts \cite{misc:dsr:mayring}. 
% \paragraph*{Other Methods:}
% Design requirements can also be generated using other methods or combinations of methods.
% Other methods include simulations, experiments, case studies, ethnography, and the grounded theory approach \cite{misc:dsr:nickerson, misc:dsr:varshney}. \\\\

\paragraph{\ac{dr}1: Heterogeneous Users} states that \textit{the approach should support diverse users with different needs, goals, and capabilities and integrate internal and external users.} 
It is supported by literature indicating that different users may have different requirements, preferences, and levels of technical expertise \cite{misc:lean:steve}.
By including a diverse group of users, you can get a broader range of feedback and insights into how the software performs for different users \cite{article:prototyping:weichbroth} and reduces the biases among developers \cite{misc:lean:burmeister}.
In this context, the users can be from internal sources, such as employees, or external sources, like Amazon Mechanical Turk\footnote{Website of Amazon Mechanical Turk: \url{https://www.mturk.com/}} for using the software.

\paragraph{\ac{dr}2: Iterative Design} states that \textit{the approach should have an iterative, incremental method and identify and address any technical issues or design flaws early in the development process.} 
It is supported by literature indicating that the iterative approach involves breaking down development into small, incremental work cycles rather than delivering a complete product all at once \cite{misc:lean:tutorial}.
The key benefit of an iterative approach is that it allows the development team to get feedback from users and stakeholders early in the development process and to make adjustments \cite{article:experiments:lindgren} to the product as needed.

\paragraph{\ac{dr}3: Easy Development} states that \textit{the approach should be easy to develop and operatable by non-technical individuals with different techniques to create applications without extensive programming knowledge.}
It is supported by literature indicating that a tool should have a UI that helps non-technical individuals to build software without including the developers. 
It can be achieved if the tool provides drag-and-drop interfaces \cite{article:nocode:miller}, reusable pre-built components \cite{article:prototyping:lowcode} (e.g., buttons, textbox, and other \ac{ui} components), and a logical flow (E.g., \textit{Screen1} is followed by \textit{Screen2} and so on.). 
We saw these features in many tools like \texttt{Figma\footnote{Website of Figma Prototyping tool: \url{https://www.figma.com/}}}, \texttt{Invision\footnote{Website of Invision: \url{https://www.invisionapp.com/}}}, and \texttt{Axure\footnote{Website of Axure Rapid Prototyping: \url{https://www.axure.com/}}}.

\paragraph{\ac{dr}4: Integrate Data Models} states that \textit{the approach should easily integrate data models (incorporating \ac{crud} operations of data models) and iterate them using various \ac{ui} elements on the screens.} 
It is supported by literature indicating that by using the tool, citizen developers can easily access and integrate data models from multiple sources, including databases, \ac{api}s, and external systems \cite{paper:lowcode:khorram}, without having to write complex code or build custom integrations from scratch \cite{article:lowcode:modeldriven}.
It accelerates the development process, reduces the risk of errors \cite{misc:lowcode:platforms}, and improves the software tool's overall quality.
In many tools like \texttt{Figma}, \texttt{Invision}, and \texttt{Axure}, we saw the use of data models\footnote{These tools either have their implementation or depend on some third-party data model tool.}.

\paragraph{\ac{dr}5: Classified UI variants} states that \textit{the approach should create multiple \ac{ui} variants or versions, each with distinct UI elements or features.}
It is supported by literature indicating different users may have other preferences regarding how they interact with an application \cite{article:swdemand:ahmed}.
At the same time, we saw many tools like Google Optimize\footnote{Website of Google Optimize: \url{https://marketingplatform.google.com/about/optimize/}}, VWO\footnote{Website of VWO: \url{https://vwo.com/de/}}, Convertize\footnote{Website of Convertize: \url{https://www.convertize.com/}}, and Freshmarketer\footnote{Website of Freshmarketer: \url{https://www.freshworks.com/crm/marketing/}} having provisions to create various \ac{ui} variants or versions.

\paragraph{\ac{dr}6: Conduct UI Split tests} states that \textit{the approach should conduct various \ac{ui} split tests on the participants using different \ac{ui} variants or versions.}
It is supported by literature indicating that a product can be tested against different design solutions and variations \cite{article:CE:fitzgerald} to see which variant performs the best in terms of usability, aesthetics, and other characteristics \cite{article:controlled:experiements}. 
At the same time, continuously improve \cite{article:CE:ros} the product based on the feedback of the best-fit variant.

\paragraph{\ac{dr}7: Construct User Scenarios} states that \textit{the approach should observe and record how users interact with a software tool to evaluate the tool's ease of use.} 
It is supported by literature indicating that testing of the \ac{gui} of a software application is done using functional and usability tests \cite{misc:usability:tasks}. 
This helps the developers to identify any usability issues \cite{article:tbup:kari} and improve them continuously \cite{article:prototyping:gould}.
And this helps identify and validate user requirements in the early stages of development \cite{article:prototyping:weichbroth}. 


\paragraph{\ac{dr}8: Collect Feedback} states that \textit{the approach should gather various user feedbacks (such as user behavior patterns, click rates or open-ended questions) from the split tests.}
It is supported by literature indicating that feedback can be collected while observing the participants performing the tasks \cite{misc:qualitative:qualitative}, like asking open-ended questions about their overall experience.
At the same time, it should automatically record any feedback that participants give and analyze it while looking for some pattern in the data \cite{article:qqa:young}.

\paragraph{\ac{dr}9: Aggregated Feedback} states that \textit{the approach should collect the gathered feedback and aggregate them to make improvements to the application.} 
It is supported by literature indicating that qualitative analysis gathers an in-depth understanding of underlying reasons, opinions, and motivations \cite{misc:dsr:mayring}.
Whereas quantitative analysis measures and understands numerical data and helps identify patterns and trends \cite{article:qqa:young}.
An aggregation of the qualitative and quantitative analysis can provide a more complete picture of a situation and can be used to validate or disprove findings from one type of analysis \cite{article:qq:helena}.

\paragraph{\ac{dr}10: Improvement} states that \textit{the approach should improve the prototypes from the results of the collected feedback in an iterative manner.} 
It is supported by literature indicating that visualization helps in prototyping by allowing the users to see and understand the design in a way that is easy to understand \cite{article:comparative:prototypes}.
It also allows users to identify usability issues \cite{article:prototyping:gould} early in the design process to make the end product user-friendly and easy to use by improving the prototype.
In tools, various methods, like creating \textit{Graphs}, \textit{Charts}, \textit{Plots}, etc. are used for visualization for improving the products.


% \paragraph*{\ac{dr}10: Provide feedback assistance} states that \textit{A Software product should help the citizen developers with the analysis by guiding them on what questions should be asked to the participants.}
% It is supported by literature indicating that there are many ways of creating qualitative analysis questions \cite{misc:dsr:mayring} confusing the citizen developers on deciding the correct approach \cite{misc:qualitative:qualitative}.
% Therefore, there should be a recommendation system for suggesting the relevant questions called post-task questionnaires.
% Some examples\footnote{Examples post-task questionnaires: \url{https://uxpsychology.substack.com/p/standardized-usability-questionnaires-ccb}} are ASQ\footnote{ASQ (After Scenario Questions): \url{https://ehealth.uvic.ca/resources/tools/UsabilityBenchmarking/05a-2011.02.15-ASQ_and_PSSUQ_Questionnaires-no_supplements-v3.0.pdf}} (After Scenario Questionnaire) which consists of 3 questions, SMEQ (Subjective Mental Effort Questionnaire) which is composed of 1 question, etc. 

\clearpage
%********************************** % Design Principles **************************************
\section{Design Principles}
\label{design:section:designprinciple}
\ac{dp}s are guidelines or rules used to guide the design process in \ac{dsr} \cite{misc:dsr:henver}. 
They provide a framework for making design decisions \cite{paper:designprinciple:gregor} and help to ensure that the final solution meets the goals and objectives of the research. 
This section codifies our knowledge during the design study and derives \ac{dp}s from abstract \ac{dr}s in the iteration cycle of the \ac{dsr}.
The following shows the \textit{nine \ac{dp}s} for \textit{our solution approach} that is built on the foundation of the mapped \ac{dr}s (see figure \ref{fig:design:table-drs-dps}). 
\begin{figure}[htbp!]
  \centering    
  \includegraphics[width=0.85\textwidth]{Table-drs-dps.png}
  \caption[A Map Between DRs and DPs]{A Map Between DRs and DPs}
  \label{fig:design:table-drs-dps}
\end{figure}
\clearpage

\paragraph{\ac{dp}1: Modeling:} \textit{The solution approach is required to provide techniques for incorporating models that can be used to simplify and visualize the \ac{ui} prototype so that developers can test its functionality and identify potential issues before the actual software is developed.}

Modeling increases transparency among various users, increasing their contribution to the product due to their excellent visualization and improvement capability (i.e., \textit{DR10: Continuous Improvement}).
Furthermore, modeling tools can automatically generate code or other documentation from the models, which can help reduce errors, improve efficiency, and decrease development time iteratively (i.e., \textit{DR2: Iterative Design}).

\paragraph{\ac{dp}2: User Variety:} \textit{The solution approach is required to provide techniques for incorporating diverse types of users, both the prototype creators, i.e., \ac{ux} designers, and participants, i.e., users who participate in an experiment so that developers can engage with various users during the evaluation process.}

Developers have many unclear generalities early in the product development process \cite{misc:lean:steve} that they can clarify by testing the underlying assumptions using different types of users (i.e., \textit{DR1: Heterogenous Users}) and involving various product users using an iterative design for continuous improvement (i.e., \textit{DR2: Iterative Design}).
This helps to gather the user requirements smoothly, thus improving the product's usability.

\paragraph{\ac{dp}3: Flexible \ac{ui} Elements:} \textit{The solution approach is required to provide techniques for incorporating a library of \ac{ui} elements and interactive components that can be easily customized and used in the prototype so that developers can demonstrate how the \ac{ui} will function.}

A \ac{ui} Prototyping tool is helpful when different interactive (i.e., \textit{DR3: Easy Development}) re-usable components are used while creating the prototypes and integrating various data models (i.e., \textit{DR4: Integrate Data Models}).
These elements can be pre-designed \ac{ui} elements such as buttons, menus, forms, etc., and interactive components such as textboxes, checkboxes, etc.
It helps to get more feedback from the users or the participants, improving the product's usability and functionality.

\paragraph{\ac{dp}4: User tasks Refinement:} \textit{The solution approach is required to provide techniques for incorporating task creation that simulates real-world scenarios and workflows and assigns randomly to participants so that developers can collect and improve the \ac{ui} prototype with the help of data.}

We can improve the usability of software by observing different users (i.e., \textit{DR1: Heterogenous Users}) as they interact with the product and measuring how well they can accomplish specific tasks or goals (i.e., \textit{DR7: Construct User Scenarios}).
These tasks are some real-world scenarios and workflows, with clear instructions and defined success criteria.
Moreover, the users can provide valuable insights into how easy or difficult it is to use the product and help identify areas where developers could improve the \ac{ui} or design aggregating user feedback (i.e., \textit{DR9: Aggregated Feedback}).

\paragraph{\ac{dp}5: Split Tests:} \textit{The solution approach is required to provide techniques for incorporating the creation of different versions of \ac{ui} prototypes and conduct tests to compare the performance of each version so that developers can make data-driven decisions about which \ac{ui} design works best for their users.}

Split Tests (e.g., A/B tests) are performed on a randomly divided sample group of different users (i.e., \textit{DR1: Heterogenous Users}) by exposing each group to the \ac{ui} of different versions (i.e., \textit{DR5: Classified UI variants} \& \textit{DR6: Conduct Split tests}) to find out the feature that is most usable and functional for the users.
The test results are then used to determine which version is more effective by aggregating the feedback (i.e., \textit{DR9: Aggregated Feedback}) from the user groups and finally optimizing the whole product.

% In many tools, we saw the use of many \ac{desy} like \texttt{Design System Components} (e.g., buttons, forms etc.), \texttt{Design System Patterns} (e.g., patterns for communication and data transfer between components), \texttt{Accessibility standards} (e.g., some guidelines to promote accessibility).

\paragraph{\ac{dp}6: Qualitative Analysis:} \textit{The solution approach is required to provide techniques for incorporating and analyzing qualitative feedback from participants so that developers can gain insights into the user experience and identify areas for improvement.}

We can conduct qualitative analysis systematically for examining non-numerical data to uncover patterns, themes, and insights from different users (i.e., \textit{DR1: Heterogenous Users}). 
It can be achieved through various methods (i.e., \textit{DR8: Collect Feedback}), such as content analysis or user feedback analysis, which involve coding, categorization, and interpretation of the data.

\paragraph{\ac{dp}7: Quantitative Analysis:} \textit{The solution approach is required to provide techniques for incorporating quantitative analysis features so that developers can analyze and visualize data from A/B testing and other analytics for improving the \ac{ui} prototype.}

We can conduct quantitative analysis systematically to test hypotheses, measure relationships between variables, and make statistical inferences about populations based on representative samples from different users (i.e., \textit{DR1: Heterogenous Users}).
It can be achieved by assigning tasks to various users in the study (i.e., \textit{DR7: Construct User Scenarios}) and collecting their feedback (i.e., \textit{DR8: Collect Feedback}) for their particular \ac{ui} variant in the split test (i.e., \textit{DR6: Conduct split tests}).

\paragraph{\ac{dp}8: Accumulated Analytics:} \textit{The solution approach is required to provide techniques for incorporating different analytics and metrics so that developers can track the performance and behavior of the various \ac{ui} prototype variants being tested.}

Performing various tests (e.g., usability testing) with diverse users (i.e., \textit{DR1: Heterogenous Users}) ensures that the software is accessible and easy to use for different groups of people with an accurate evaluation of the task provided to the users (i.e., \textit{DR7: Construct User Scenarios}).
Moreover, diversity in software development feedback mechanisms (i.e., \textit{DR6: Collect Feedback}), received by testing different UI versions (i.e., \textit{DR5: Classified UI variants}), helps ensure the software is inclusive and accessible.

\paragraph{\ac{dp}9: Continuous Integration:} \textit{The solution approach is required to provide techniques for incorporating the tool's design into small, incremental phases in the software development process so that developers can improve software delivery and make product changes and improvements based on customer feedback.}

Using a continuous incremental approach, we can continuously improve software products by delivering value to customers as quickly as possible, constantly refining and improving the product, and delivering the product as soon as possible.
Here, the iterative design should be used (i.e., \textit{DR2: Iterative Design}) to get continuous feedback from various users (i.e., \textit{DR1: Heterogenous Users}).
The collected feedback should be a combination (i.e., \textit{DR7: Aggregate Feedback}) of various feedback, helping significantly improve the application.

\clearpage
%********************************** % Solution Concept **************************************
\section{Overall Solution Design}
\label{design:section:solutiondesign}
As shown in figure \ref{fig:design:lean}, we conceptualize the solution design from our codified \ac{dp}s.
The software platform consists of two types of roles consisting of the users for creating the Prototyping tool (\textit{\ac{ux} Designers}) and the \textit{Users or Participants} for testing the tool.
In this section, we arrange our \ac{dp}s using a LEAN development approach\footnote{Website for the LEAN development: \url{https://www.lean.org/explore-lean/product-process-development/}}, a cycle consisting of \textit{Build}, \textit{Measure}, \textit{Learn} phases.

\begin{figure}[htbp!]
  \centering    
  \includegraphics[width=0.85\textwidth]{LEAN-DPs.png}
  \caption[Solution Concept]{Overall Solution Design for a UI Prototyping tool using LEAN principles\footnotemark[10]}
  \label{fig:design:lean}
\end{figure}

In the LEAN development cycle, the whole process is done in a model-based approach (i.e., \ac{dp}1) to improve the solution approach.
%********************************** % Build **************************************** 
\paragraph{Build:}
\label{design:paragraph:build}
Initially, different users register to our tool (i.e., \ac{dp}2) having different rights. 
The \ac{ux} designers build a \ac{ui} prototype using various reusable \ac{ui} components and create data models (i.e., \ac{dp}3). 
Using the \ac{ui} elements, these users make various screens and connect them to various data models to have a logical flow of the prototype.
In the next step, they create various tasks simulating real-world scenarios, assign them randomly to participants and collect feedback (i.e., \ac{dp}4).
After creating the tasks, the \ac{ux} designers create various \ac{ui} variants or versions for creating split tests (i.e., \ac{dp}5).

%********************************** % Measure **************************************
\paragraph{Measure:}
\label{design:paragraph:measure}
In this phase, we measure the data we receive from the tasks by conducting qualitative (i.e., \ac{dp}6) and quantitative (i.e., \ac{dp}7) analysis.
The quantitative analysis is done from the data the users collect from the tasks.
Similarly, the qualitative analysis is done by collecting the comments, open-ended questions etc. after the task is finished. 
%********************************** % Learn ****************************************
\paragraph{Learn:}
\label{design:paragraph:learn}
In this phase, the data received from the analysis is processed and visualized. 
The data is processed by aggregating and combining the results of the qualitative and the quantitative analysis (i.e., \ac{dp}8). 
Finally, the models are given feedback from the data analysis, improving the UI prototype and updating it with the winner variant (i.e., \ac{dp}9) as the default \ac{ui}.
The refined product can be deployed for the entire population (or all the users) using the deployment module using a no-code development approach.
