%!TEX root = ../../thesis.tex
%*******************************************************************************
%****************************** Design Chapter *********************************
%*******************************************************************************

\chapter{Solution Design}

\ifpdf
    \graphicspath{{Chapters/Design/Figs/}{Chapters/Design/Figs/}{Chapters/Design/Figs/}}
\else
    \graphicspath{{Chapters/Design/Figs/}{Chapters/Design/Figs/}}
\fi
Design is a critical aspect of \ac{se}, providing the blueprint for implementing functional and non-functional requirements. 
The design phase is crucial in ensuring that software systems are developed efficiently, effectively, and with high quality. 
In this chapter, we will explore the \ac{dr} (see section \ref{design:section:designReqs}) and \ac{dp} (see section \ref{design:section:designprinciple}) that guide the creation of our software tool.
We develop our design approach based on the principles of \ac{dsr}.
We also explore different design architectures (see section \ref{design:section:architecture}) and meta-modeling techniques (see section \ref{design:section:metaModeling}) to create efficient, scalable, and maintainable software systems. 
Through this exploration, we will provide an in-depth understanding of the design process, highlighting the key factors that must be considered to create effective software solutions.

%********************************** % Design Requirements **************************************
\section{Design Requirements (DRs)}
\label{design:section:designReqs}
\ac{dr}s in \ac{dsr} are typically defined as a set of constraints and specifications that must be met by the design artifact to be considered a successful solution. 
These requirements can be functional, such as the specific features or capabilities that the design/tool must have, or non-functional, such as performance, usability, or scalability.
In our findings of the \ac{dr}s, we focus more on the functional requirements and ignore the non-functional.
To derive a solution, we define some \ac{dr}s with the help of the literature review\footnote{It was a non-systematic literature review by reading research papers and looking at some renowned UI prototyping tools. The related literature is available in Appendix \ref{appendix:two:definations}} and comparison of some tools.
In this context, each \ac{dr} refers to a generalized requirement that can be standardized and applied to future software applications.
We covered a wide range of topics including \textit{\ac{ui} Prototyping}, \textit{Low-Code/No-Code development}, \textit{Model-Based Software Engineering}, \textit{Continuous Experimentation}, \textit{Task-Based Usability Testing}, and the \textit{LEAN development process}.
The following section presents \textit{eleven \ac{dr}s} and their corresponding references in literature and tools.
% \paragraph*{Systematic literature reviews:} 
% Systematic literature reviews establish a solid basis for knowledge creation. 
% They help create theories and identify areas that require more research.
% By identifying existing literature and highlighting areas to explore, they also assist in the understanding of the problem space in DSR.
% Therefore, a comprehensive DSR project must include systematic literature reviews \cite{misc:dsr:webster}.
% \paragraph*{Interviews:}
% Interviews are regarded as empirical research to obtain primarily qualitative insights from individuals, such as specialists in a particular subject. 
% Suppose the interviewers are a part of the problem's stakeholder group.
% In that case, they may be utilized to establish requirements (or meta requirements) for the solution space that have a solid foundation for the problem space.
% For example, artifacts can be designed based on design principles (DPs) generated from interviews.
% Furthermore, the interviews can also be used to improve and evaluate the designed artifacts \cite{misc:dsr:mayring}. 
% \paragraph*{Other Methods:}
% Design requirements can also be generated using other methods or combinations of methods.
% Other methods include simulations, experiments, case studies, ethnography, and the grounded theory approach \cite{misc:dsr:nickerson, misc:dsr:varshney}. \\\\

\paragraph{\ac{dr}1: Heterogeneous Users} states that \textit{the tool should support diverse users with different needs, goals, and capabilities and integrate internal and external users.} 
It is supported by literature indicating that different users may have different needs, preferences, and levels of technical expertise \cite{misc:lean:steve}.
By including a diverse group of users, you can get a broader range of feedback and insights into how the software performs for different users \cite{article:prototyping:weichbroth}, and reduces the biases among developers \cite{misc:lean:burmeister}.
In this context, the users can be from internal sources, such as employees, or external sources, like Amazon Mechanical Turk\footnote{Amazon Mechanical Turk: \url{https://www.mturk.com/}} for using the software.

\paragraph{\ac{dr}2: Iterative Design} states that \textit{the tool should be designed in an iterative, incremental approach with some continuous feedback mechanism.} 
It is supported by literature indicating that the iterative approach involves the idea of breaking down development into small, incremental cycles of work rather than trying to deliver a complete product all at once \cite{misc:lean:tutorial}.
The key benefit of an iterative approach is that it allows the development team to get feedback from users and stakeholders early in the development process and to make adjustments \cite{article:experiments:lindgren} to the product as needed.

\paragraph{\ac{dr}3: Easy Development} states that \textit{the tool should be easy to develop and operatable by non-technical individuals with drag-and-drop interfaces and pre-built components to create applications without extensive programming knowledge.} 
It is supported by literature indicating that a tool should have a UI that helps non-technical individuals to build software without including the developers. 
It can be achieved if the tool provides drag-and-drop interfaces \cite{article:nocode:miller}, reusable pre-built components \cite{article:prototyping:lowcode} (e.g., buttons, textbox, and other \ac{ui} components, data models for data transfer between components) and a logical flow (E.g., \textit{Screen1} is followed by \textit{Screen2} and so on.). 
In many tools like \texttt{Figma\footnote{Figma Prototyping tool: \url{https://www.figma.com/}}}, \texttt{Invision\footnote{Invision: \url{https://www.invisionapp.com/}}}, and \texttt{Axure\footnote{Axure Rapid Prototyping: \url{https://www.axure.com/}}}, we saw these features.

\paragraph{\ac{dr}4: Integrate Data Models} states that \textit{the tool should be designed to make it easier to integrate data models by providing pre-built connectors and APIs that can be used to connect to various data sources.} 
It is supported by literature indicating that by using the tool, citizen developers can easily access and integrate data models from multiple sources, including databases, APIs, and external systems \cite{paper:lowcode:khorram}, without having to write complex code or build custom integrations from scratch \cite{article:lowcode:modeldriven}.
It accelerates the development process, reduces the risk of errors \cite{misc:lowcode:platforms}, and improves the software tool's overall quality.
In many tools like \texttt{Figma}, \texttt{Invision}, and \texttt{Axure}, we saw the use of data models\footnote{These tools either have their own implementation or depend on some third-party data model tool.}.

\paragraph{\ac{dr}5: Engage Stakeholders} states that \textit{Various stakeholders (e.g., designers, product managers, developers, etc.) must contribute to the quick development of the product.}
It is supported by literature indicating that involving different project stakeholders and providing them with the necessary communication tools helps exchange and codify knowledge \cite{article:prototyping:weichbroth}. 
This makes sure that the stakeholders' perspectives, requirements \cite{misc:prorotypes:lauff}, and their input helps shape the product's design.

% \paragraph{\ac{dr}5: Diverse Re-usable components} states that \textit{Collaboration between different team members and usage of the reusable components of various disciplines should make the tool easy for anyone to use on any platform and share their work quickly.} 
% It is supported by literature indicating developers' ability to create more consistent, user-friendly prototypes that adhere to the re-usable components' established style and principles \cite{paper:prototyping:luqi} and help improve the overall design process and product development \cite{article:prototyping:hoffnagle}.
% In many tools, we saw the use of many reusable components like buttons, textbox, and other \ac{ui} components, data models for data transfer between components, and some guidelines to promote accessibility and usability of the tools.

\paragraph{\ac{dr}6: Enhanced Evaluation} states that \textit{The software product should be evaluated using various means by having users perform specific tasks, collect feedback, and improve the product.} 
It is supported by literature indicating that testing of the \ac{gui} of a software application is done using functional and usability tests \cite{misc:usability:tasks}. 
This helps the developers to identify any usability issues \cite{article:tbup:kari} and improve them continuously \cite{article:prototyping:gould}.
And this helps in the identification and preliminary validation of user requirements in the early stages of development \cite{article:prototyping:weichbroth}. 

% \paragraph{\ac{dr}7: Improved Accessibility} states that \textit{The tool should be a web-based application making it accessible and independent of any software, platforms (e.g., Macbooks, Windows \ac{pc}s, Linux machines, Chromebooks) and have an auto-save feature to store the work on Cloud.}
% It is supported by literature indicating that Web-based tools have several advantages \cite{misc:cloud} over traditional application-based tools.
% In many tools like \texttt{Figma}, \texttt{Invision}, and \texttt{Axure}, we saw features\footnote{Some more features: \url{https://redrocksoftware.com.au/10-benefits-of-web-based-applications-systems/}} like \texttt{Accessibility} i.e., can be accessed from any device with an internet connection, \texttt{Collaboration} i.e., have built-in collaboration features, making it easy for team members to work on the same prototype simultaneously, \texttt{Updates} i.e., automatically updated.

% \textbf{DR9: Get user feedback} \textit{} Having users perform specific tasks and observe how they interact with the application helps improve the usability of the software applications. So, we can use task-based usability testing for achieving this. 
\paragraph{\ac{dr}7: Aggregated feedback} states that \textit{The software product should gather user feedback and aggregate them to make improvements to the application.} 
It is supported by literature indicating that \texttt{Qualitative analysis} gathers an in-depth understanding of underlying reasons, opinions, and motivations \cite{misc:dsr:mayring}.
Whereas \texttt{Quantitative analysis} measures and understands numerical data and helps identify patterns and trends \cite{article:qqa:young}.
And together, qualitative and quantitative analysis can provide a more complete picture of a situation and can be used to validate or disprove findings from one type of analysis \cite{article:qq:helena}.

\paragraph{\ac{dr}8: Visualization} states that \textit{There should be a visualization tool for prototyping to help different stakeholders see and interact with the design and gather feedback.} 
It is supported by literature indicating that visualization helps in prototyping by allowing designers and stakeholders to see and understand the design in a way that is easy to understand \cite{article:comparative:prototypes}.
It also allows designers to identify usability issues \cite{article:prototyping:gould} early in the design process to make the end product user-friendly and easy to use.
In tools, various methods, like creating \texttt{Wireframes}, \texttt{Mockups}, and \texttt{Interactive prototypes}, are used for visualization.

% \paragraph{\ac{dr}10 Quick delivery} states that \textit{A software product should be easily deliverable, reducing the complexity and deployment time and increasing the product's usability.} 
% It is supported by literature indicating that when software is easy to build, development teams can be more productive and efficient, leading to lower development costs \cite{misc:lowcode:platforms} and can be quickly delivered. 
% It's also easy for different development team members to understand and work on the code, thereby improving collaboration \cite{misc:prorotypes:lauff} and generating high-quality code.
% Adding new features, scaling up, and making changes, as needed, make the tool more accessible and helpful \cite{article:prototyping:lowcode} to ensure that the product is developed and deployed with minimum effort \cite{article:prototyping:lowcode}.

\paragraph*{\ac{dr}9: Classified UI versions} states that \textit{A Software product should be tested with various \ac{ui} versions or variants to analyze the best-fit variant for the product.}
It is supported by literature indicating that a product can be tested against different design solutions and variations \cite{article:CE:fitzgerald} to see which variant performs the best in terms of usability, aesthetics, and other characteristics \cite{article:controlled:experiements}. 
At the same time, continuously improve \cite{article:CE:ros} the product based on the feedback of the best-fit variant.

\paragraph*{\ac{dr}10: Provide feedback assistance} states that \textit{A Software product should help the citizen developers with the analysis by guiding them on what questions should be asked to the participants.}
It is supported by literature indicating that there are many ways of creating qualitative analysis questions \cite{misc:dsr:mayring} confusing the citizen developers on deciding the correct approach \cite{misc:qualitative:qualitative}.
Therefore, there should be a recommendation system for suggesting the relevant questions called post-task questionnaires.
Some examples\footnote{Examples post-task questionnaires: \url{https://uxpsychology.substack.com/p/standardized-usability-questionnaires-ccb}} are ASQ\footnote{ASQ (After Scenario Questions): \url{https://ehealth.uvic.ca/resources/tools/UsabilityBenchmarking/05a-2011.02.15-ASQ_and_PSSUQ_Questionnaires-no_supplements-v3.0.pdf}} (After Scenario Questionnaire) which consists of 3 questions, SMEQ (Subjective Mental Effort Questionnaire) which is composed of 1 question, etc. 

\clearpage
%********************************** % Design Principles **************************************
\subsection{Design Principles (DPs)}
\label{design:section:designprinciple}
Design Principles (DPs) are guidelines or rules used to guide the design process in \ac{dsr} \cite{misc:dsr:henver}. 
They provide a framework for making design decisions \cite{paper:designprinciple:gregor} and help to ensure that the final solution meets the goals and objectives of the research. 
\ac{dp}s are used for breaking down complex problems into smaller, more manageable parts.
This section codifies our knowledge during the design study and derives \ac{dp}s from abstract \ac{dr}s from the iteration cycle of the \ac{dsr}.
The following shows the \textit{Eight \ac{dp}s} and references to literature and tools that build the foundation for the mapped \ac{dr}s (see figure \ref{fig:design:table-drs-dps}). 
\begin{figure}[htbp!]
  \centering    
  \includegraphics[width=0.85\textwidth]{Table-drs-dps.png}
  \caption[A map between \ac{dr}s and \ac{dp}s]{A map between Design Requirements (\ac{dr}s) and Design Principles (\ac{dp}s)}
  \label{fig:design:table-drs-dps}
\end{figure}

\paragraph{\ac{dp}1: User Variety:} \textit{Users with diverse requirements, objectives, and abilities should be able to use the product, and it should be able to accommodate both internal and external users within the organization.}

Developers have many unclear generalities early in the product development process \cite{misc:lean:steve} that they can clarify by testing the underlying assumptions using different types of users (i.e., \textit{DR1: Heterogenous Users}), and involving various product stakeholders (i.e., \textit{DR4: Engage Stakeholders}).
This helps to gather the user requirements smoothly, thus improving the product's usability.

\paragraph{\ac{dp}2: Split Tests:} \textit{Split tests can improve a software product by continuously testing different \ac{ui} versions or variants with shuffled users to get a winner or a best-fit variant from the users' feedback and improve the product.}

Split Tests (also known as A/B Testing) are performed on a randomly divided sample group of different users (i.e., \textit{DR1: Heterogenous Users}) by exposing each group to the \ac{ui} of different versions (i.e., \textit{DR11: Classified UI Versions}) to find out the feature that is most usable and functional for the users.
The results of the test are then used to determine which version is more effective by combining the feedback (i.e., \textit{DR8: Combined Feedback}) from the user groups and finally optimizing the whole product interatively (i.e., \textit{DR2: Iterative Refinement}).

\paragraph{\ac{dp}3: Continuous Incremental Design:} \textit{A software product broken down into minor, incremental phases in the software development process helps quickly improve software delivery and also helps to make product changes and improvements based on customer feedback.}

Using a continuous incremental approach, we can continuously improve software products by delivering value to customers as quickly as possible \cite{misc:lean:toyota}, constantly refining and improving the product \cite{misc:lean:planning}, and delivering the product as soon as possible.
Here, the iterative design should be used (i.e, \textit{DR2: Iterative Refinement}) to get continuous feedback from a variety of users (i.e., \textit{DR1: Heterogenous Users}).
The feedback which is collected should be a combination (i.e., \textit{DR8: Combined Feedback}) of various feedback helping significantly improve the application.

\paragraph{\ac{dp}4: Varied Design Systems:} \textit{A product should be developed using Design Systems innovatively by the citizen developers without complete reliance and dependence on the \ac{it} department.}

A \texttt{\ac{desy}}\footnote{Design Systems: \url{https://www.invisionapp.com/inside-design/guide-to-design-systems/}} is a collection of reusable components governed by some standards that can be assembled in many ways to make different applications.
A \ac{ui} Prototyping tool is helpful when different interactive (i.e., \textit{DR3: Interactive Design}) re-usable components are used as a drag-and-drop element (i.e., \textit{DR5: Diverse Re-usable Components}) while creating the prototypes by engaging various stakeholders' contributions (i.e., \textit{DR4: Engage Stakeholders}). 
It helps to get more feedback (i.e., \textit{DR8: Combined Feedback}) from the users or the customers, improving the product's usability and Functionality.
% In many tools, we saw the use of many \ac{desy} like \texttt{Design System Components} (e.g., buttons, forms etc.), \texttt{Design System Patterns} (e.g., patterns for communication and data transfer between components), \texttt{Accessibility standards} (e.g., some guidelines to promote accessibility).

\paragraph{\ac{dp}5: Task-based Usability Tests:} \textit{Usability testing can provide valuable insights into how users interact with a product and can help to improve the \ac{ux}. It's an effective method to detect usability issues early in the development process by assigning tasks to the participants and ensuring the product is user-friendly, efficient, and easy to use, utilizing the feedback from the tasks.}

We can improve the usability of software by observing different users (i.e., \textit{DR1: Heterogenous Users}) as they interact with the product and measuring how well they can accomplish specific tasks or goals (i.e., \textit{DR6: Enhanced Evaluation}).
Moreover, the users can provide valuable insights into how easy or difficult it is to use the product and help identify areas where developers could improve the UI or design using user feedback (i.e., \textit{DR8: Combined Feedback}). 

\paragraph{\ac{dp}6: Feedback Diversity:} \textit{Diversity in software development feedback mechanisms helps ensure that a wide range of perspectives and experiences are considered when designing and testing software.}

Performing various tests (e.g., usability testing) with diverse users (i.e., \textit{DR1: Heterogenous Users}) ensures that the software is accessible and easy to use for different groups of people with an accurate evaluation of the task provided to the users (i.e., \textit{DR6: Enhanced Evaluation}).
Moreover, diversity in software development feedback mechanisms (i.e., \textit{DR8: Combined Feedback}), received by testing different UI versions (i.e., \textit{DR11: Classified UI Versions}), helps ensure the software is inclusive and accessible.

\paragraph{\ac{dp}7: Modeling for Verification} \textit{Modeling provides a clear and concise representation of the system being developed, thereby increasing communication among team members and stakeholders.}

A model-based approach increases transparency among various stakeholders, increasing their contribution to the product (i.e., \textit{DR4: Engage Stakeholders}) due to their excellent visualization capability (i.e., \textit{DR9: Visualization}).
Furthermore, modeling tools can automatically generate code or other documentation from the models, which can help reduce errors, improve efficiency, and decrease development time (i.e., \textit{DR10: Quick Delivery}).

\paragraph{\ac{dp} 8: Rapid Solution Delivery:} \textit{Rapid solution delivery in software development refers to the quick development and deployment of software solutions that meet the needs of users and stakeholders.}

We can achieve rapid solution delivery through the use of agile development methodologies and iterative design of the software (i.e., \textit{DR2: Iterative Refinement}).
With rapid solution delivery, software development teams can get feedback from users and stakeholders more quickly (i.e., \textit{DR10: Quick Delivery}) and adjust from the feedback increasing accessibility (i.e., \textit{DR7: Improved Accessibility}).

\paragraph{\ac{dp}9: Feedback assistance for qualitative research:} \textit{By providing guidelines on what type of questions (rating based or open-ended) to ask of the participants, the software product should aid citizen developers with qualitative analysis.}

We can provide a feedback assistance system (i.e., \textit{\ac{dr}12: Assistance for Feedback}) for qualitative analysis helping the citizen developers formulate a qualitative questionnaire. 
These questions can be asked to the participants either during the tasks or after the completion of the task (i.e., \textit{\ac{dr}6: Enhanced Evaluation}) as a post-survey questionnaire. 
Thus, having a predefined qualitative questionnaire helps to automate the feedback mechanism (i.e., \textit{\ac{dr}8: Combined Feedback}).

\clearpage
%********************************** % Solution Concept **************************************
\section{Solution Concept}
\label{design:section:solutionconcept}
As shown in figure \ref{fig:design:lean}, we conceptualize the solution design from our codified \ac{dp}s.
The software platform consists of two types of roles consisting of the Stakeholders for creating the Prototyping tool (E.g., \textit{Product Owners}, the \textit{Designers}) and the \textit{Users or Participants} for testing the tool.
We plan to design our tool using a LEAN development approach, a cycle consisting of \textit{Build} (see section \ref{design:section:build}), \textit{Measure} (see section \ref{design:section:measure}), \textit{Learn} (see section \ref{design:section:learn}) steps and continuous improvement throughout the process.
% \begin{figure}[htbp!]
%   \centering    
%   \includegraphics[width=0.9\textwidth]{Sol-design.png}
%   \caption[Solution Design]{Solution Design for a UI Prototyping tool}
%   \label{fig:design:solutiondesign}
% \end{figure}

\begin{figure}[htbp!]
  \centering    
  \includegraphics[width=1\textwidth]{LEAN-DPs.png}
  \caption[Solution Concept]{Solution Concept for a UI Prototyping tool using LEAN development}
  \label{fig:design:lean}
\end{figure}

%********************************** % Build **************************************
\subsection{Build}
\label{design:section:build}

%********************************** % Measure **************************************
\subsection{Measure}
\label{design:section:measure}
%********************************** % Learn **************************************
\subsection{Learn}
\label{design:section:learn}

\section{Architecture}
\label{design:section:architecture}

\section{Meta-Modeling}
\label{design:section:metaModeling}