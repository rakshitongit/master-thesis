%!TEX root = ../thesis.tex
%*******************************************************************************
%****************************** Evaluation Chapter *********************************
%*******************************************************************************

\chapter{Evaluation}
\label{chap:evaluation}
In the previous chapter, we explained the implementation of our tool. 
Based on that, in this chapter, we will discuss the details of the evaluation process, including the user case study (see section \ref{evaluation:section:casestudy}), experimental design (see section \ref{evaluation:section:design}), execution (see section \ref{evaluation:section:execution}), analysis (see section \ref{evaluation:section:analysis}), and interpretation (see section \ref{evaluation:section:interpretation}) of the results. 
For the setup, we recruited participants from Paderborn University.
\ifpdf
    \graphicspath{{Chapters/Evaluation/Figs/}{Chapters/Evaluation/Figs/}{Chapters/Evaluation/Figs/}}
\else
    \graphicspath{{Chapters/Evaluation/Figs/}{Chapters/Evaluation/Figs/}}
\fi

\section{User Case Study}
\label{evaluation:section:casestudy}
To evaluate the effectiveness of our approach, we conducted a case study that involved recruiting participants, developing prototypes, and working on user scenarios. 
We recruited 15 participants who are students at Paderborn University. 
The case study is based on the evaluation stage of the first cycle of our DSR (as discussed in section \ref{introduction:section:research}). 
In conducting and reporting our case study research, we followed the established guidelines of Runeson and Höst \cite{eval:guidlines:runeson} to increase the quality of the study outcomes. 
These guidelines helped ensure our research was rigorous, transparent, and credible. 
By adhering to these guidelines, we aimed to provide a detailed and comprehensive account of our case study, enabling others to replicate our research and build upon our findings.
This case study also aims to address the RQ developed in the introduction of this thesis.

\section{Experimental Design}
\label{evaluation:section:design}
For the User Case Study, we aimed to recruit diverse participants from different courses enrolled at Paderborn University to ensure a wide range of perspectives and experiences on UI prototyping and UI Experimentation. 
We wanted to include individuals with varying levels of experience in prototyping and user scenario development, from beginners to experts. 
To recruit participants, we used Doodle\footnote{Website for Doodle: \url{https://doodle.com/}}, an online tool for setting up the appointment, and a Line survey\footnote{Website for the survey: \url{https://umfragen.uni-paderborn.de/admin/}} hosted by our university for creating the questionnaire.

We developed a user scenario that required participants to use our UI prototyping tool hosted on the Paderborn University server. 
Therefore to access our tool, the participants needed to turn on the VPN\footnote{Website for University Paderborn VPN config: \url{https://imt.uni-paderborn.de/vpn-zugang/}} provided by the University. 
We also examined them using an open BigBlueButton\footnote{Website for BBB: \url{https://open-bbb.uni-paderborn.de/}} video conference session. 
We also considered the ethical clarifications by informing the participants that we were not recording the video conference session and that the survey was anonymous to ensure their privacy and encourage honest feedback while mentioning our data collection and storage strategy.

\section{Execution}
\label{evaluation:section:execution}
\section{Analysis}
\label{evaluation:section:analysis}
\section{Interpretation}
\label{evaluation:section:interpretation}