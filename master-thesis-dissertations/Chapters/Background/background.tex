%!TEX root = ../thesis.tex
%*******************************************************************************
%****************************** Background Chapter *********************************
%*******************************************************************************

\chapter{Background}
\ifpdf
    \graphicspath{{Chapters/Background/Figs/}{Chapters/Background/Figs/}{Chapters/Background/Figs/}}
\else
    \graphicspath{{Chapters/Background/Figs/}{Chapters/Background/Figs/}}
\fi
To build the foundation of our approach, we present the usage of crowdsourcing in software development (see section \ref{background:section:crowdsourcing}), UI Prototyping (see section \ref{background:section:uiprototyping}), Low and No code (see section \ref{background:section:lowcode}), Model Based Software Engineering (MBSE) (see section \ref{background:section:mbse}), Task Based Usability Testing (see section \ref{background:section:task}), and Experimental Product Design and define Design Principles (DP) (see section \ref{background:section:experimentproduct}).

%********************************** % Crowdsourcing of Software Products **************************************
\section{Crowdsourcing of Software Products}
\label{background:section:crowdsourcing}
Iterative feedback from potential customers can help improve the development of software products \cite{article:lean:eric}.
To do that, we can use crowdsourcing.
Crowdsourcing refers to outsourcing value-creating activities from a company by an open call to a large, undefined group of users to get feedback \cite{article:crowdsourcing:leimeister}.
The word crowdsourcing is a combination of crowd and outsourcing.
Crowdsourcing often involves less specialized and more generalized groups of participants than outsourcing \cite{article:crowdsourcing:estelles}.
Some advantages of crowdsourcing include lowered costs, improved speed, quality, flexibility, and scalability \cite{article:crowdsourcing:prpic}.
Researchers have used crowdsourcing in many research approaches, including \textit{crowd testing, crowd funding, crowd ideation, crowd logistic, crowd production, crowd promotion,} and \textit{crowd support} over the last few years \cite{article:crowdsourcing:durward}.
In our approach to finding a solution, we focus more on crowd-testing and crowd-ideation.

\paragraph{Crowd Testing:}
The companies use crowd-testing to evaluate different running software products with the users.
A growing trend in software testing is crowd testing, which utilizes the benefits, effectiveness, and efficiency of crowdsourcing and cloud platforms \cite{article:crowdsourcing:latoza}.
Crowd testing is considered when the software is more user-centric: i.e., software with a broad user base whose success is evaluated by user input.
CrowdStudy \cite{article:crowdsourcing:nebeling} is a method that enables developers to assess the usability of their web interfaces using crowd workers from Amazon Mechanical Turk\footnote{Amazon Mechanical Turk: \url{https://www.mturk.com}}.
CrowdCrit \cite{article:crowdsourcing:luther} is another tool that uses Amazon Mechanical Turk to support designers in validating created posters in the form of uploaded images.
Similarly, \textit{Interactive event-flow graphs} and \textit{GUI-level (Graphical User Interface) guidance} \cite{article:crowdsourcing:chen} are the two techniques to increase crowd testers' coverage for GUI using crowd-testing.

\paragraph{Crowd Ideation:}
Design can be infused with creativity by online crowds, but using traditional strategies to harness them, such as large-scale ideation platforms, requires organization and time \cite{article:crowdsourcing:andolina}.
Hence, crowd ideation is used to build new and improved versions of existing software product ideas with the consumers.
Under manipulations of task complexity, idea representation, and procedural guidance, Shixuan Fu et al. \cite{article:crowdsourcing:fu} examine how cognitive load is altered during idea generation and convergence with crowds.
ERICA \cite{article:crowdsourcing:erica} is a tool that uses expert knowledge to validate diverse crowd answers.
Crowdboard \cite{article:crowdsourcing:andolina} is a tool used to engage crowds in real-time brainstorming, concept mapping, and other design processes at an early stage of the design process.
There were, however, no approaches that directly addressed prototype application areas.

%********************************** % UI Prototyping **************************************
\section{UI Prototyping}
\label{background:section:uiprototyping}
User Interface prototyping is an evaluation and testing technique according to User-Centred Design (UCD) methodology since the 1990s \cite{article:prototyping:preece}. 
The evaluation of prototypes by users is a fundamental part of all iterative approaches for IT project management, especially agile methodologies \cite{article:prototyping:schwaber}.
And to build an exemplary user interface, iterative refinement must be used: develop a preliminary version of the user interface, test it with people, and make as many revisions as possible \cite{article:prototyping:gould}.
Therefore, designing UI prototypes enables designers and stakeholders to communicate more effectively.
An interactive prototype helps visualize design concepts and communicate new requirements and expectations about a prospective system.
Iterative design requires multiple updates to the design's execution.
Since developing and updating the entire software system is complex and expensive, prototyping is a crucial technique \cite{article:prototyping:szekely}.
Simultaneously, software prototypes might exclude many requirements, making the software more accessible, smaller, and less expensive to construct and change \cite{article:prototyping:szekely}. 
Similarly, usability testing to validate user requirements and prototype functionality is part of the evaluation process for UI prototypes.
When prototyping is used, there is usually more contact between the designers and users, resulting in fewer usability flaws and corrections at the end of development.

Jim Rudd et al. \cite{article:prototyping:highlowfidelity} have compared high and low-fidelity prototyping, explaining the advantages and disadvantages.
\textit{Low-fidelity} prototypes are usually limited function, with little interaction prototyping effort. They mainly focus on explaining concepts, design alternatives, and screen layouts. 
Storyboard presentations, cards, and proof of concept prototypes come under this category.
These prototypes emphasize communicating, educating, and informing rather than training, testing, and codification.
The advantages of low-fidelity prototypes are rapid development, lower development cost, addressing issues, and usefulness for a proof-of-concept.
Similarly, the disadvantages include limited error checking, difficulty with usability testing, navigation, flow limitation, etc.
Contrary to low-fidelity prototypes, \textit{High-fidelity} prototypes have full functionality and focus on flow, and the user models of the system \cite{article:prototyping:exploratory}.
The users can operate these prototypes, and the developers can collect information from the users through measurements. 
Other advantages of high-fidelity prototypes are that they are user-driven, used for navigation and tests, and can also be served as a marketing tool for attracting potential customers \cite{article:prototyping:highlowfidelity}.

% \begin{figure}[htbp!]
% \centering    
% \includegraphics[width=1.0\textwidth]{minion.png}
% \caption[Minion]{This is just a long figure caption for the minion in Despicable Me from Pixar}
% \label{fig:minion}
% \end{figure}

%********************************** % Low code **************************************
\section{Low Code / No Code}
\label{background:section:lowcode}

%********************************** % Model-Based Software Engineering **************************************
\section{Model-based Software Engineering}
\label{background:section:mbse}
Model-based Software Engineering (MBSE) refers to maintaining and developing software while reusing existing code.
Similarly, Model-driven software engineering (MDSE) is the term used to cover various techniques for creating software using codified models.
The development of domain-specific languages (DSLs) is becoming essential in language engineering due to the growth in model-driven engineering (MDE) \cite{article:mbse:cuadrado}.
MDSE has become an integral part of developing User interfaces, and they have been named Model-driven User Interfaces (MUIs).
Based on that, adaptive model-driven user interface development systems are developed \cite{article:mbse:akiki}.
In this research, the authors defined twenty properties challenges for the Model-driven User interface and compared some tools that implement these properties.

Companies use different modeling languages to codify the UIs.
Cameleon \cite{article:cameleon:balme} is a framework that divides the UI into several elements to maximize the parts' reusability in various user, platform, and environment situations.
A platform-independent abstract UI, a platform-dependent concrete UI, and a device-dependent final UI are the layers the framework offers to accomplish this.
A standardized modeling language for software product content, abstract UI models, user interactions, and control behavior is Interaction Flow Modeling Language (IFML) \cite{article:ifml:piero}.
As a result, IFML relies on the platform-independent display of the UI that can be utilized on several platforms and devices.

However, these modeling languages do not emphasize offering visual notations to aid non-developers in creating such interfaces. 
A recent method \cite{article:mbse:bexiga} illustrates how to use low-code approaches to close the gap between designers and developers.

%********************************** % Task-based Usability Testing **************************************
\section{Task-based Usability Testing}
\label{background:section:task}

%********************************** % Experimental Product Design **************************************
\section{Experimental Product Design}
\label{background:section:experimentproduct}

\section*{Description}
\begin{description}
\item[The first topic] is dull
\item[The second topic] is duller
\begin{description}
\item[The first subtopic] is silly
\item[The second subtopic] is stupid
\end{description}
\item[The third topic] is the dullest
\end{description}

% \clearpage

\begin{landscape}

\section{Landscape}
I can cite Wall-E (see Fig.~\ref{fig:WallE}) and Minions in despicable me (Fig.~\ref{fig:Minnion}) or I can cite the whole figure as Fig.~\ref{fig:animations}


\begin{figure}
  \centering
  \begin{subfigure}[b]{0.3\textwidth}
    \includegraphics[width=\textwidth]{minion.png}
    \caption{Tom and Jerry}
    \label{fig:TomJerry}   
  \end{subfigure}             
  \begin{subfigure}[b]{0.3\textwidth}
    \includegraphics[width=\textwidth]{minion.png}
    \caption{Wall-E}
    \label{fig:WallE}
  \end{subfigure}             
  \begin{subfigure}[b]{0.3\textwidth}
    \includegraphics[width=\textwidth]{minion}
    \caption{Minions}
    \label{fig:Minnion}
  \end{subfigure}
  \caption{Best Animations}
  \label{fig:animations}
\end{figure}

\end{landscape}
