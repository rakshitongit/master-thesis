%!TEX root = ../thesis.tex
%*******************************************************************************
%****************************** Background Chapter *********************************
%*******************************************************************************

\chapter{Background}
\ifpdf
    \graphicspath{{Chapters/Background/Figs/}{Chapters/Background/Figs/}{Chapters/Background/Figs/}}
\else
    \graphicspath{{Chapters/Background/Figs/}{Chapters/Background/Figs/}}
\fi
This chapter makes the reader familiar with some concepts needed for better understading the thesis by including UI Prototyping (see section \ref{background:section:uiprototyping}), Low and No code (see section \ref{background:section:lowcode}), Model Based Software Engineering (MBSE) (see section \ref{background:section:mbse}), Task Based Usability Testing (see section \ref{background:section:task}), Experimental Product Design (see section \ref{background:section:experimentproduct}), and Crowdsourcing (see section \ref{background:section:crowdsourcing}).

%********************************** % UI Prototyping **************************************
\section{UI Prototyping}
\label{background:section:uiprototyping}

\begin{figure}[htbp!] 
\centering    
\includegraphics[width=1.0\textwidth]{minion.png}
\caption[Minion]{This is just a long figure caption for the minion in Despicable Me from Pixar}
\label{fig:minion}
\end{figure}

%********************************** % Low code **************************************
\section{Low Code / No Code}
\label{background:section:lowcode}

%********************************** % Model-Based Software Engineering **************************************
\section{Model-based Software Engineering}
\label{background:section:mbse}

\begin{enumerate}
\item The first topic is dull
\item The second topic is duller
\begin{enumerate}
\item The first subtopic is silly
\item The second subtopic is stupid
\end{enumerate}
\item The third topic is the dullest
\end{enumerate}

%********************************** % Task based Usability Testing **************************************
\section{Task-based Usability Testing}
\label{background:section:task}

%********************************** % Experimental Product Design **************************************
\section{Experimental Product Design}
\label{background:section:experimentproduct}

%********************************** % Crowdsourcing **************************************
\section{Crowdsourcing}
\label{background:section:crowdsourcing}

\begin{itemize}
\item The first topic is dull
\item The second topic is duller
\begin{itemize}
\item The first subtopic is silly
\item The second subtopic is stupid
\end{itemize}
\item The third topic is the dullest
\end{itemize}

\section*{Description}
\begin{description}
\item[The first topic] is dull
\item[The second topic] is duller
\begin{description}
\item[The first subtopic] is silly
\item[The second subtopic] is stupid
\end{description}
\item[The third topic] is the dullest
\end{description}

% \clearpage

\begin{landscape}

\section{Landscape}
I can cite Wall-E (see Fig.~\ref{fig:WallE}) and Minions in despicable me (Fig.~\ref{fig:Minnion}) or I can cite the whole figure as Fig.~\ref{fig:animations}


\begin{figure}
  \centering
  \begin{subfigure}[b]{0.3\textwidth}
    \includegraphics[width=\textwidth]{minion.png}
    \caption{Tom and Jerry}
    \label{fig:TomJerry}   
  \end{subfigure}             
  \begin{subfigure}[b]{0.3\textwidth}
    \includegraphics[width=\textwidth]{minion.png}
    \caption{Wall-E}
    \label{fig:WallE}
  \end{subfigure}             
  \begin{subfigure}[b]{0.3\textwidth}
    \includegraphics[width=\textwidth]{minion}
    \caption{Minions}
    \label{fig:Minnion}
  \end{subfigure}
  \caption{Best Animations}
  \label{fig:animations}
\end{figure}

\end{landscape}
