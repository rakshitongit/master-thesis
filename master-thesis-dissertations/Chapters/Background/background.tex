%!TEX root = ../../thesis.tex
%*******************************************************************************
%****************************** Background Chapter *********************************
%*******************************************************************************

\chapter{Background}
\ifpdf
    \graphicspath{{Chapters/Background/Figs/}{Chapters/Background/Figs/}{Chapters/Background/Figs/}}
\else
    \graphicspath{{Chapters/Background/Figs/}{Chapters/Background/Figs/}}
\fi
To build the foundation of our approach, we present UI Prototyping (see section \ref{background:section:uiprototyping}), Low and No code (see section \ref{background:section:lowcode}), Model-Based Software Engineering (MBSE) (see section \ref{background:section:mbse}), Task-Based Usability Testing (see section \ref{background:section:task}), and Experimental Product Design (see section \ref{background:section:experimentproduct}) and derive some Design Requirements (see section \ref{background:section:designReqs}).
%********************************** % Crowdsourcing of Software Products **************************************
% \section{Crowdsourcing of Software Products}
% \label{background:section:crowdsourcing}
% ***************************** \textbf{Optional section} ***************************** \\\\
% Iterative feedback from potential customers can help improve the development of software products \cite{article:lean:eric}.
% To do that, we can use crowdsourcing.
% Crowdsourcing refers to outsourcing value-creating activities from a company by an open call to a large, undefined group of users to get feedback \cite{article:crowdsourcing:leimeister}.
% The word crowdsourcing is a combination of crowd and outsourcing.
% Crowdsourcing often involves less specialized and more generalized groups of participants than outsourcing \cite{article:crowdsourcing:estelles}.
% Some advantages of crowdsourcing include lowered costs, improved speed, quality, flexibility, and scalability \cite{article:crowdsourcing:prpic}.
% Researchers have used crowdsourcing in many research approaches, including \textit{crowd testing, crowd funding, crowd ideation, crowd logistic, crowd production, crowd promotion,} and \textit{crowd support} over the last few years \cite{article:crowdsourcing:durward}.
% In our approach to finding a solution, we focus more on crowd-testing and crowd-ideation.

% \paragraph{Crowd Testing:}
% The companies use crowd-testing to evaluate different running software products with the users.
% A growing trend in software testing is crowd testing, which utilizes the benefits, effectiveness, and efficiency of crowdsourcing and cloud platforms \cite{article:crowdsourcing:latoza}.
% Crowd testing is considered when the software is more user-centric: i.e., software with a broad user base whose success is evaluated by user input.
% CrowdStudy \cite{article:crowdsourcing:nebeling} is a method that enables developers to assess the usability of their web interfaces using crowd workers from Amazon Mechanical Turk\footnote{Amazon Mechanical Turk: \url{https://www.mturk.com}}.
% CrowdCrit \cite{article:crowdsourcing:luther} is another tool that uses Amazon Mechanical Turk to support designers in validating created posters in the form of uploaded images.
% Similarly, \textit{Interactive event-flow graphs} and \textit{GUI-level (Graphical User Interface) guidance} \cite{article:crowdsourcing:chen} are the two techniques to increase crowd testers' coverage for GUI using crowd-testing.

% \paragraph{Crowd Ideation:}
% Design can be infused with creativity by online crowds, but using traditional strategies to harness them, such as large-scale ideation platforms, requires organization and time \cite{article:crowdsourcing:andolina}.
% Hence, crowd ideation is used to build new and improved versions of existing software product ideas with the consumers.
% Under manipulations of task complexity, idea representation, and procedural guidance, Shixuan Fu et al. \cite{article:crowdsourcing:fu} examine how cognitive load is altered during idea generation and convergence with crowds.
% ERICA \cite{article:crowdsourcing:erica} is a tool that uses expert knowledge to validate diverse crowd answers.
% Crowdboard \cite{article:crowdsourcing:andolina} is a tool used to engage crowds in real-time brainstorming, concept mapping, and other design processes at an early stage of the design process.
% There were, however, no approaches that directly addressed prototype application areas.
% \clearpage
%********************************** % UI Prototyping **************************************
\section{UI Prototyping}
\label{background:section:uiprototyping}
\begin{figure}[htbp!]
  \centering    
  \includegraphics[width=0.6\textwidth]{PrototypingSteps.png}
  \caption[Steps of Prototyping]{Prototyping steps for an iterative development}
  \label{fig:background:stepsPrototyping}
\end{figure}
User Interface (UI) prototyping is an evaluation and testing technique according to User-Centred Design (UCD) methodology since the 1990s \cite{article:prototyping:preece}.
A UI prototype is a mock-up of the User interface.
Before a real product is created, prototypes are used to test the usability and user experience of the interface.
The evaluation of prototypes by users is a fundamental part of all iterative approaches for IT project management, especially agile methodologies \cite{article:prototyping:schwaber}.
And to build an exemplary user interface, a company must use iterative refinement: develop a preliminary version of the user interface, test it with people, and make as many revisions as possible \cite{article:prototyping:gould}.
Figure \ref{fig:background:stepsPrototyping} shows a cycle that can be used for the iterative development of prototypes.
The designers start the process by developing the UI prototypes, which stakeholders (e.g., customers and product managers) review. The UI prototypes are refined from the feedback received, reiterating the cycle.
Therefore, designing UI prototypes enables designers and stakeholders to communicate more effectively.

An interactive prototype helps visualize design concepts and communicate new requirements and expectations about a prospective system.
Iterative design requires multiple updates to the design's execution.
\begin{figure}[htbp!]
  \centering
  \begin{subfigure}[b]{0.6\textwidth}
    \includegraphics[width=\textwidth]{paperprototyping.png}
    \caption{Sketces and whiteboard for prototyping in initial stages of development}
    \label{fig:background:paperPrototyping}   
  \end{subfigure}             
  \begin{subfigure}[b]{0.5\textwidth}
    \includegraphics[width=\textwidth]{paperstepsprototypes.png}
    \caption[Paper based prototypes]{Example on how the paper-based prototypes are developed  \cite{misc:prototyping:uxpin}}
    \label{fig:background:paperprototypes}
  \end{subfigure}             
  \caption[Low fidelity prototyping]{Low fidelity prototyping}
  \label{fig:background:main}
\end{figure}
Since developing and updating the entire software system is complex and expensive, prototyping is a crucial technique \cite{article:prototyping:szekely}.
Simultaneously, software prototypes might exclude many requirements, making the software more accessible, smaller, and less expensive to construct and change \cite{article:prototyping:szekely}. 
Similarly, usability testing to validate user requirements and prototype functionality is part of the evaluation process for UI prototypes.
When prototyping is used, there is usually more contact between the designers and users, resulting in fewer usability flaws and corrections at the end of development.
The main difference between a prototype and a software application is that in a prototype, the displays are designed images with no additional capabilities to display the design and flow. The mockups are converted into actual UI elements in a software application, and a flow is available.

From a paper to HTML code, everything could be a prototype.
Jim Rudd et al. \cite{article:prototyping:highlowfidelity} have compared high and low-fidelity prototyping, explaining the advantages and disadvantages.
\textit{Low-fidelity} prototypes (see figure \ref{fig:background:paperPrototyping}) are usually limited functions with little interaction prototyping effort. They mainly focus on explaining concepts, design alternatives, and screen layouts. 
Storyboard presentations, cards, and proof of concept prototypes come under this category.
UI designers use simple text, lines, and forms to hand-draw concepts. 
Instead of aesthetics, the focus is on speed and many ideas.
To simulate user flows (as shown by figure \ref{fig:background:paperprototypes}), designers lay paper screens on the floor, table, or pinned to a board.
These prototypes emphasize communicating, educating, and informing rather than training, testing, and codification.
The advantages of low-fidelity prototypes are rapid development, lower development cost, addressing issues, and usefulness for a proof-of-concept.
Similarly, the disadvantages include limited error checking, difficulty with usability testing, navigation, flow limitation, etc.
\begin{figure}[htbp!]
  \centering    
  \includegraphics[width=0.9\textwidth]{ui-based-prototyping.png}
  \caption[High Fidelity prototyping]{High Fidelity prototype: Model-based UI Prototyping}
  \label{fig:background:uiPrototyping}
\end{figure}

Contrary to low-fidelity prototypes, \textit{High-fidelity} prototypes (see figure \ref{fig:background:uiPrototyping}) have full functionality and focus on flow, and the user models of the system \cite{article:prototyping:exploratory}.
The users can operate these prototypes, and the developers can collect information from the users through measurements. 
Other advantages of high-fidelity prototypes are that they are user-driven, used for navigation and tests, and can also be served as a marketing tool for attracting potential customers \cite{article:prototyping:highlowfidelity}.\\ \\
\textit{Steps for creating UI Prototypes}:
\begin{itemize}
  \item \textbf{Learn about your consumers' requirements:} Here, our primary goal is to understand the principles that underlie our client's values to better our service to suit their needs. In this step, the design team meets with the product management team and brainstorms to develop some use cases and end-user needs. After the discussions, the team can present the customers to ensure they are on the right track and that the design matches their requirements.
  \item \textbf{Sketch out the product:} This is an important step for UI designers. In this step, the UI designers can develop low-fidelity prototypes, e.g., The design team can create the paper prototypes and start brainstorming different possible options. Further, the team can create high-fidelity prototypes by using some tools. There are some tools available like Figma\footnote{Figma: \url{https://www.figma.com/}}, Invision\footnote{Invion AG \url{https://www.invisionapp.com/}}, Adobe XD\footnote{Adobe XD: \url{https://www.adobe.com/products/xd.html}}, Axure RP\footnote{Axure RP: \url{https://www.axure.com/}} and many more. Using these tools, the designer can develop some high-fidelity digital prototypes.
  \item \textbf{Develop into Software:} This step includes the software developers and the designers coordinating and developing the software in iterations. The designers would give feedback to the software developers, and they would be continuously developing the product.
\end{itemize}

After performing these steps, it is necessary to test if everything is working as expected by the user requirements.
It is necessary to ensure that the product is usable for the end-users and that developers get all the essential features in the evaluation phase. 
\clearpage
%********************************** % Low code **************************************
\section{Low Code / No Code Development Platform}
\label{background:section:lowcode}
Low Code is a technique used by developers to help non-developers design and develop software applications using a \textit{Graphical User Interface} (GUI) supported by a \textit{Low Code Development Platform} (LCDP).
Similarly, there is another technique called no code supported by the \textit{No Code Development Platform} (NCDP) \cite{article:nocode:miller}.
Unlike low code, no-code platforms require no programming skills because they offer some pre-build templates for building the apps.
Using the visual user interface and ready-made automatic tools on these application development platforms, it is feasible to create apps relatively quickly. 
These technologies are visual app development methods that speed up app creation using drag-and-drop editors and pre-built components. 
By allowing software developers to concentrate on challenging coding areas, low code / no code accelerates platform development.
Due to its simplicity, flexibility, and low cost, companies have started using this platform to meet the high demands of software development and digitalization.
Low code is a software development method that uses less human coding to enable users to construct and manage programs efficiently \cite{article:nocode:sahina}.
Additionally, it lowers the expenses associated with initial installation, training, distribution, and maintenance \cite{article:nocode:sanchi}.\\ \\
\textit{Main Steps Of \textbf{Low-Code} App Development}
\paragraph*{Building:}
\begin{figure}[htbp!]
  \centering    
  \includegraphics[width=0.7\textwidth]{step1-building.png}
  \caption[Building]{Step 1: Building of Low Code App development}
  \label{fig:background:building}
\end{figure}
In this step (as shown in the figure \ref{fig:background:building}), the platform gives you the freedom to alter the provided code and add hand-written custom code to it to specify more complex features in the app as you create the app step-by-step using visual editors and drag and drop interfaces.
Modules, components, and chart-builders are already incorporated into low-code applications. 
Charts may be used to display data from modules, while modules are used to specify the type of data that will be stored in the app. 
Components and pages provide the type of user experience the app will have.
These platforms also have a provision for automating repetitive tasks in the app.
\paragraph*{Testing:}
\begin{figure}[htbp!]
  \centering    
  \includegraphics[width=0.9\textwidth]{step2-testing.png}
  \caption[Testing]{Step 2: Testing of Low Code App development}
  \label{fig:background:testing}
\end{figure}
Testing a software application is an integral part of the development cycle.
However, the low-code development platform decreases the requirement for testing. 
Pre-build modules and components on low-code platforms are created with a certain level of application security. 
The developers of the low-code platform constantly monitor these modules, and they have previously gone through several unit tests.
But (as shown in figure \ref{fig:background:testing}), we still need to perform the test on the entire application after integration for scrutiny, which is performed in this step.
\paragraph*{Deploying:}
\begin{figure}[htbp!]
  \centering    
  \includegraphics[width=0.7\textwidth]{step3-deployment.png}
  \caption[Testing]{Step 3: Deployment of Low Code App development}
  \label{fig:background:deploying}
\end{figure}
In this step, the application is deployed across apps and to the final users.
In LCDP, the packages for installation, configurations, and application setup are included.
And the app's deployment can be done on various services (see figure \ref{fig:background:deploying}) like Cloud-based services.
This means that the LCDP gives freedom to the users in deploying the applications for the customers with just one click.  \\\\
\textit{Some features that make the LCDP or NCDP favorable for development} \cite{article:nocode:sahina, article:nocode:ihirwe, paper:lowcode:cabot}
\begin{itemize}
  \item \textbf{Re-usable components:} LCDP usually contains pre-configured components, modules, logic, templates, and many more that can be customized per customer requirements. This helps to Build apps with consistency and scalability. Also, the scope for testing decreases as the components of the LCDP is pre-tested for security and performance. 
  \item \textbf{Flexible and Model-driven development:} The drag-and-drop functionality helps developers increase productivity, whereas citizen developers build all apps. LCDP ensures to have a GUI useful for the non-developers. Therefore, it encourages the stakeholders outside the IT department to participate in development. Low code has become famous in model-driven development. The model-driven product helps to visualize how the app works while it is built.
  \item \textbf{Easy deployment:} This feature ensures that the artifacts are created, and the platform is ready to deploy. LCDP has packages that contain various deployment packages (e.g., \texttt{Dockerfile} to deploy on docker\footnote{Docker: \url{https://www.docker.com/}})
\end{itemize}

Additionally, a variety of options are provided for developers with little programming experience, those with coding expertise and seasoned programmers who wish to expand the functionality of the current design \cite{article:nocode:sahina}.
\clearpage
%********************************** % Model-Based Software Engineering **************************************
\section{Model-based Software Engineering}
\label{background:section:mbse}
\begin{figure}[htbp!]
  \centering    
  \includegraphics[width=0.5\textwidth]{mof-meta-modelling.png}
  \caption[MOF levels]{Model Object Facility (MOF) levels}
  \label{fig:background:moflevels}
\end{figure}

Model-based Software Engineering (MBSE) refers to maintaining and developing software while reusing existing code.
Similarly, Model-driven software engineering (MDSE) is the term used to cover various techniques for creating software using codified models.
At the same time, for creating models, the Model Object Facility (MOF) has defined four levels separating the reality, models, meta-models, and meta-meta-models (as shown in the figure \ref{fig:background:moflevels}).
\begin{figure}[htbp!]
  \centering    
  \includegraphics[width=0.9\textwidth]{meta-models.png}
  \caption[Meta models]{Meta models and Models}
  \label{fig:background:m1m2}
\end{figure}

\paragraph{Meta Models:} 
A meta-model is defined as a model of a model or a simplified version of an actual model of a system of interest or a software application.
They can be used to understand how a model works or to improve the performance of a model by making it more efficient or accurate. 
In software development, meta-models are models used to represent a software system's structure and behavior.
Meta models are often used in model-driven development, focusing on creating and using high-level abstractions to design and implement software systems. By using meta-models, developers can create a high-level view of a system and then use that view to generate code automatically, which can help improve the development process's efficiency and reliability.
Meta models can act as an intermediary between the input and the output relations and represent mathematical relations or algorithms.
A model is usually defined as an abstraction of real-world entities, and a meta-model is another abstraction of the models. 
Similarly, metamodelling is analyzing and developing the rules, theories, and models helpful in constructing meta-models.
From figure \ref{fig:background:m1m2}, the meta-models are very similar to the UML class diagrams, are used to define the models or grammar of the models, and the models are instances of them. 
The meta-model is technically described using its metamodeling structures which is self-describing. 
This offers a consistent semantic approach for Model Driven Architecture (MDA) artifacts representing models and meta-models.

\paragraph{Models: } 
As shown in figure \ref{fig:background:moflevels}, a \textit{Model} is a simplified representation of the real world or an unavoidable reality.
Before any code is written, the model is a schematic that describes how the software system should function.
In software development, a model is a representation of a software system. 
It can be used to visualize the structure and behavior of a system or to analyze the system for potential issues or improvements. 
Models can take many forms, such as diagrams, graphs, or mathematical equations, and they can be created using various tools and techniques. 
Models are often used in model-driven development, focusing on creating and using high-level abstractions to design and implement software systems. 
By using models, developers can better understand a system and identify potential problems or areas for improvement before they start writing code.
From the example \ref{fig:background:m1m2}, we need to create models for every entity of reality (e.g., An employee and a CEO both are \textit{Person} in meta-model but are separate entities in models).
Using models, various adaptive model-driven user interface development systems are developed \cite{article:mbse:akiki}.
In this research, the authors defined twenty properties challenges for the Model-driven User interface and compared some tools that implement these properties.
Similarly, modeling is a process and method of building models for some purpose or related to some domain.
Therefore, modeling languages are used to codify the UIs in different companies. 
% Cameleon \cite{article:cameleon:balme} is a framework that divides the UI into several elements to maximize the parts' reusability in various user, platform, and environment situations.
% A platform-independent abstract UI, a platform-dependent concrete UI, and a device-dependent final UI are the layers the framework offers to accomplish this.

However, these modeling languages do not emphasize offering visual notations to aid non-developers in creating such interfaces. 
For our research, we use a recent method \cite{article:mbse:bexiga}, which illustrates how to use low-code and Model-driven approaches to close the gap between designers and developers.

\clearpage
%********************************** % Task-based Usability Testing **************************************
\section{Task-based Usability Testing}
\label{background:section:task}
The main focus of usability testing is that seeing someone use an interface is the best approach to determine what functions well and doesn't. 
It would help if you offered the participants some assignments to observe them. 
The word "task" is commonly used to describe these assignments.
Assigning tasks to the accurate number of participants can help determine the quality of the UI and the problems faced by the users. 
Overall, the UI design can be improved using the participants' feedback. 
Task-based usability testing is one way to determine the software's overall usability \cite{article:usability:doesburg} by measuring the percentage of the tasks the users complete.
To observe the participants, they need to be assigned some ``activities'' or \textit{tasks}. 
These tasks need to be some scenarios, not just "\textit{do something}", because it sets the users a stage for \textit{why} they would perform the tasks. 
To get qualitative feedback from the participants, in \cite{misc:usability:tasks}, the authors provide \textit{three good practices} and task-writing tips for designing better task scenarios.

\paragraph{\textit{(1) Make the Task Realistic}:}
So, the participants should be able to execute the tasks which could be completed efficiently and with the freedom to make their own choices.
The participants will attempt to accomplish the assignment without genuinely interacting with the interface if you ask them to do something they wouldn't typically do. 
Therefore, it is necessary to create realistic tasks. \\\\
E.g., from \textit{Videostreamer} application: The goal is to offer some movies that the user should watch. \\
\textbf{Bad task: } Watch a movie `ABC movie'.\\
\textbf{Good Task: } Watch a movie with more than 6.5/10 ratings. \\

\paragraph{\textit{(2) Make the Task Actionable}:}
Here, the participants should be told what they need to do rather than how they would do it.
These types of tasks help us determine if the task isn’t actionable enough. 
E.g., if the participants tell the moderator, they cannot determine if they need to click on the link or if they are finding it hard to decide the next steps. \\
From \textit{Videostreamer} application: The goal is to find a movie and show times. \\\\
\textbf{Bad Task: } You want to see a movie Sunday afternoon. Go to the app and tell where you’d click next. \\
\textbf{Good Task: } Use the app to find a movie you’d be interested in seeing on Sunday afternoon. \\

\paragraph{\textit{(3) Avoid Giving Clues and Describing the Steps}:}
There are frequent hints about how to use the interface or the software in step explanations.
These tasks must be more balanced with the users' behavior and give less valuable results.   
The participants should expose the navigation and some features on their own, giving accurate feedback about the interface.
But, at the same time, we should try to include the words used in the UIs as they help the users navigate smoothly and would not lead to some confusion. \\\\
From \textit{Videostreamer} application: The goal is to change the user's movie preferences. \\
\textbf{Bad Task: } You want to change your movie preferences. Go to the website, sign in, and change your movie preferences. \\
\textbf{Good Task: } Change your movie preferences.

\begin{figure}[htbp!]
  \centering    
  \includegraphics[width=0.5\textwidth]{task-based-cycle.png}
  \caption[Tasks steps]{Task-based Usability Testing steps}
  \label{fig:background:taskssteps}
\end{figure}
The fundamental principle of usability testing is to have actual users attempt to carry out essential tasks or assignments on your software, websites, mobile applications, or gadgets.
For that, we divide the tasks into three activities.
\paragraph{Planning: } 
In usability testing, planning is an important step. 
In this step, we decide on a process that fits our research question, hypothesis, and metrics.  
It is beneficial if we use mixed-methods services for a task-based usability study.
The tasks and scenarios must be well-defined in the planning phase. At the same time, we can create pre-study and post-study questions. 
\paragraph{Recruitment: } 
In this step, the users are assigned to the task as it is essential to select the correct number of user group sizes.
There should be a proper way to interview the users before assigning them tasks so that the participants understand them correctly. 
The participants should also be able to ask questions or doubts while performing the honorarium tasks. 
\paragraph{Analysis: }
Task-based studies analyze metrics, issues, and insights in great depth.
For metrics, we calculate the task completion rates, task time, and task and test level perception questions.
The problems that the participants encounter while performing the tasks need to be reported automatically to the development teams by sending screenshots, quotes, etc. 
There should also be a section to analyze some insights about the software that has worked well for the users while performing the tasks.
\clearpage
%********************************** % Experimentation **************************************
\section{Experimentation}
\label{background:section:experimentproduct}
Experimental Product Design (EPD) has become integral to optimizing UI and \textit{User Experience (UX)}.
Experimentation helps product teams test out ideas early in the process with real-world consumers rather than settling on a single solution and executing it in the final phase \cite{misc:CE:miklos}.
Experimentation in UI prototypes refers to repeatedly testing and iterating on the user interface of a product or service over time. This approach allows designers and developers to gather user feedback, analyze the results, and make improvements to the design based on the data. 
This approach allows designers and developers to gather user feedback, analyze the results, and make improvements to the design based on the data.
It can be an effective way to improve the usability and effectiveness of a product or service.

In this section, we discuss the role that experimentation plays in the software development process and how designers can ``prototype with real data'' to improve the usability of the UI.
\begin{figure}[htbp!]
  \centering    
  \includegraphics[width=0.5\textwidth]{ab-testing.png}
  \caption[A/B Testing]{A/B Testing}
  \label{fig:background:abtesting}
\end{figure}
For conducting experiments, various steps like \textbf{Users distribution}, \textbf{Continuous Experimentation}, \textbf{Variants distribution} exist.  

\paragraph{User distribution:} To conduct a successful experiment, the size of the study, or the number of participants, must be considered first.
Statistics suggest that the more people you include in the investigation, the greater its statistical power, which impacts your level of confidence in your findings \cite{misc:experimentation:users}.
Then, the participants should be allocated into groups at random. 
There are several levels of treatment given to each group (e.g., some prerequisites to the participants before experimenting).
For assigning the subjects to groups, users are divided into a \textit{between-subjects design} vs a \textit{within-subjects design}.
In a between-subjects design, individuals receive only one of the possible levels of an experimental treatment whereas, in a within-subjects design, every individual receives each of the experimental treatments consecutively, and their responses to each treatment are measured.
In figure \ref{fig:background:abtesting}, a between-subject design is used where only one of the variants is assigned to the user.

\paragraph{Continuous Experimentation and Variants distribution:} 
Continuous experimentation (CE) primarily aims to get users' feedback on the software product's evolution.
As per figure \ref{fig:background:abtesting}, CE generally uses A/B/n testing in a primary case of comparing two variants, A and B, which are controlled and test variables in an experiment.
First, the users or the participants of the study are separated into groups and are assigned one of the two variants.
Since we have \texttt{GridView} and the \texttt{ListView}, one group of the users are assigned with List and one with Grid View (see figure \ref{fig:background:abtesting}).
Then both users would be given some tasks (see Section \ref{background:section:task}), and the analysis of these tasks gives the winner among the variants.  
And in the end, developers make evidence-based decisions to direct the progress of their software by continuously measuring the results of multiple variants performed in an experimental context with actual users \cite{article:CE:ros}.
CE is an extension to the introduction of continuous integration and deployment, and all are summarized as constant software engineering \cite{article:CE:fitzgerald}.
Additionally, continuous experimentation can help designers and developers stay up-to-date with the latest design trends and best practices. By regularly testing and iterating on the design, they can learn from the feedback they receive and incorporate new ideas and techniques into their design. 
This can help keep the product or service relevant and competitive.
\paragraph{Examples of A/B Testing:}
Some tools conduct A/B Testing and are used to improve User Interface. \textit{Optimizely}\footnote{Optimizely: \url{https://www.optimizely.com/}}, \textit{Adobe Target}\footnote{Adobe Target: \url{https://business.adobe.com/in/products/target/adobe-target.html}}, \textit{Petri}\footnote{Petri: \url{https://github.com/wix-incubator/petri}}, \textit{Google Analytics and Google Optimize}\footnote{Google Analytics and Google Optimize: \url{https://vwo.com/compare/google-optimize/}} are some of the research tools which can be used to conduct A/B Testing. \\

Overall, continuous experimentation is a critical approach to take when developing user interfaces for products and services. It allows designers and developers to gather valuable feedback and data and use it to improve their products' design and user experience.
\clearpage

\section{Design requirements}
\label{background:section:designReqs}
Prescriptive knowledge about the design of artifacts, such as software techniques, models, or concepts, is what design science research (DSR) aims can provide. 
Due to this design knowledge, future projects can design artifacts methodically and scientifically with the aid of study and practice. 
As a result, this design and application may provide design-focused information that adds to the DSR knowledge corpus \cite{misc:dsr:henver}.
Every element within a DSR project is built upon and systematically analyzed to add to the overall DSR knowledge corpus.
Several scientific methodologies are employed in the DSR to ensure that this information is presented practically using design requirements (DRs).
There are various ways to retrieve the DRs as explained below.

\paragraph*{Systematic literature reviews:} 
Systematic literature reviews establish a solid basis for knowledge creation. 
They help create theories and identify areas that require more research.
By identifying existing literature and highlighting areas to explore, they also assist in the understanding of the problem space in DSR.
Therefore, a comprehensive DSR project must include systematic literature reviews \cite{misc:dsr:webster}.

\paragraph*{Interviews:}
Interviews are regarded as empirical research to obtain primarily qualitative insights from individuals, such as specialists in a particular subject. 
Suppose the interviewers are a part of the problem's stakeholder group.
In that case, they may be utilized to establish requirements (or meta requirements) for the solution space that have a solid foundation for the problem space.
For example, artifacts can be designed based on design principles (DPs) generated from interviews.
Furthermore, the interviews can also be used to improve and evaluate the designed artifacts \cite{misc:dsr:mayring}. 

\paragraph*{Other Methods:}
Design requirements can also be generated using other methods or combinations of methods.
Other methods include simulations, experiments, case studies, ethnography, and the grounded theory approach \cite{misc:dsr:nickerson, misc:dsr:varshney}. \\ \\\\
\clearpage
For our approach, we are using the systematic literature review technique to generate the Design Requirements (DRs).
\begin{itemize}
  \item \textbf{DR1} \textit{Heterogeneous Users:} While using the prototyping, we should use different users for better research results in the research \cite{article:prototyping:weichbroth,misc:lean:steve, misc:lean:burmeister}. This is because different users may have different needs, preferences, and levels of technical expertise. By including a diverse group of users, you can get a broader range of feedback and insights into how the software performs for different users.
  \item \textbf{DR2} \textit{Different stakeholders for designing and developing software:} Involve project stakeholders by providing them with the necessary communication tools for exchanging and codifying knowledge \cite{article:prototyping:weichbroth, misc:prorotypes:lauff}. Stakeholders will have their perspectives and requirements, and their input can help shape the product's design.
  \item \textbf{DR3} \textit{Diversity in the GUI:} Do not use the commonly used GUIs for testing. This provides diversity while creating the prototypes and gathers requirements from prospective users \cite{article:prototyping:weichbroth}. CE should be added
  \item \textbf{DR4} \textit{Diversity in testing:} Testing of the GUI must be automated, and use functional and usability tests. Functional testing helps ensure the GUI works according to specification and usability testing helps ensure users can use the GUI effectively. And this helps in the identification and preliminary validation of user requirements in the early stages of development \cite{article:prototyping:weichbroth}.
  \item \textbf{DR5} \textit{Governance Mechanisms:} The solution must incorporate governance elements into the process so the platform owner can take the appropriate action against developers and users who abuse the validation process. Therefore, good governance will make sure that the users stay with the platform for a longer period \cite{misc:prototypes:evans,misc:dps:parker}. 
  \item \textbf{DR6} \textit{Different UI Properties:} The UI should be clear, concise, familiar, responsive, consistent, attractive, efficient and forgiving \cite{article:prototyping:weichbroth}. If the UI has the following properties, then it gives a better User Experience (UX).
  \item \textbf{DR7} \textit{Provide users with tasks:} Having users perform specific tasks and observing how they interact with the application helps improve the usability of the software applications. So, we can use task-based usability testing for achieving this. 
  \item \textbf{DR8} \textit{Design Models for software:} The model should be complete, consistent, and correct. Thus, modeling can help developers better understand the project's requirements, identify potential problems and design solutions, and communicate their ideas to other development team members. This helps in creating model-based prototypes with many benefits.
  \item \textbf{DR9} \textit{Diversity in feedback mechanism:} Including qualitative and quantitative feedback when evaluating a product or service is generally a good idea. Qualitative feedback is based on customers' or users' opinions and perceptions, providing insight into their experiences and feelings about the product. This type of feedback can help identify areas where the product is excelling and areas where it can be improved. Quantitative feedback, on the other hand, is based on numerical data and measurements, and it provides a more objective view of the product's performance. This type of feedback can help track trends and make comparisons over time.
  \item \textbf{DR10} \textit{Software product should be easily built:} Using Low/No-code, we can use the visual interfaces and pre-built modules to create software without writing code. This can help users build simple software applications without programming knowledge. This approach can make it easier for non-technical people to build software and automate tasks.
\end{itemize}

In conclusion, the background section of the thesis is an integral part of the overall content. 
It provides context and background information on the topic being discussed and helps the reader understand the purpose and significance of the thesis. 
This section ensures the reader has the necessary information to understand and engage with the rest of the document completely.
% \begin{landscape}

% \section{Landscape}
% I can cite Wall-E (see Fig.~\ref{fig:WallE}) and Minions in despicable me (Fig.~\ref{fig:Minnion}) or I can cite the whole figure as Fig.~\ref{fig:animations}


% \begin{figure}
%   \centering
%   \begin{subfigure}[b]{0.3\textwidth}
%     \includegraphics[width=\textwidth]{minion.png}
%     \caption{Tom and Jerry}
%     \label{fig:TomJerry}   
%   \end{subfigure}             
%   \begin{subfigure}[b]{0.3\textwidth}
%     \includegraphics[width=\textwidth]{minion.png}
%     \caption{Wall-E}
%     \label{fig:WallE}
%   \end{subfigure}             
%   \begin{subfigure}[b]{0.3\textwidth}
%     \includegraphics[width=\textwidth]{minion}
%     \caption{Minions}
%     \label{fig:Minnion}
%   \end{subfigure}
%   \caption{Best Animations}
%   \label{fig:animations}
% \end{figure}

% \end{landscape}
