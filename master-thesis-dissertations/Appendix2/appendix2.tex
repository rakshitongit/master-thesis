%!TEX root = ../thesis.tex
% ******************************* Thesis Appendix B ********************************

\chapter{Literature Review Summary}
\label{appendix:two:definations}

In this thesis, a non-systematic literature review was conducted to explore the current state of UI prototyping tools and develop the \ac{dr}s for our \ac{dsr}. 
Several research papers were reviewed, along with an examination of some renowned UI prototyping tools. 
The review focused on the features and capabilities of these tools and any potential limitations or drawbacks. 
Through this process, valuable insights were gained into the strengths and weaknesses of existing UI prototyping tools, which can inform the development of more effective tools in the future. 
This section below contains a table containing the papers (see section \ref{section:appendix:literature}) we reviewed and the tools (see section \ref{section:appendix:tools}) we tested.  

\subsection*{Software tools}
\label{section:appendix:tools}
In this section, we briefly discuss some of the tools and how they helped us formulate the \ac{dr}s. 
Our non-systematic literature review identified ten tools that could help us with our solution approach (see table \ref{table:appendix:tools:review}). 
Among these, we found five \ac{ui} prototyping tools that could be useful in designing and refining the user interface of our solution. 
These tools included Axure, Figma, Proto.io, InVision, and Adobe XD. 
Additionally, we identified five split testing tools that could be helpful in testing and improving the effectiveness of our solution. 
These split testing tools included Optimizely, VWO, Google Optimize, Freshmarketer, and Zoho PageSense. 
By incorporating these tools into our solution approach, we hope to create a user-friendly and effective solution that meets the needs of our users.
\begin{table}[htbp!]
    \centering
    \begin{tabular}{| m{0.8em} || m{13em} | m{13em} | m{4em} | }
    \hline 
    \multicolumn{4}{|c|}{\textbf{Software Tools}} \\ 
    \hline
    \textbf{\#} & \textbf{Name} & \textbf{Type} & \textbf{View} \\
    \hline
    1 & Figma &  UI Prototyping Tool & \href{https://www.figma.com/}{Link} \\
    \hline 
    2 & InVision &  UI Prototyping Tool & \href{https://www.invisionapp.com/}{Link}\\
    \hline 
    3 & Axure & UI Prototyping Tool & \href{https://www.axure.com/}{Link} \\
    \hline
    4 & Adobe XD & UI Prototyping Tool & \href{https://helpx.adobe.com/support/xd.html}{Link} \\
    \hline
    5 & Proto.io & UI Prototyping Tool & \href{https://proto.io/}{Link} \\
    \hline
    6 & Google Optimize &  Split Testing Tool & \href{https://optimize.google.com/optimize/}{Link} \\
    \hline 
    7 & VWO & Split Testing Tool & \href{https://vwo.com/blog/split-testing/}{Link} \\
    \hline 
    8 & Convertize & Split Testing Tool & \href{https://www.convertize.io/}{Link}\\ 
    \hline 
    9 & Freshmarketer & Split Testing Tool & \href{https://www.freshworks.com/crm/marketing/}{Link} \\
    \hline
    10 & Zoho PageSense & Split Testing Tool & \href{https://www.zoho.com/pagesense/}{Link} \\
    \hline
    \end{tabular}
    \caption[UI prototyping and A/B testing tools]{Table for different UI prototyping and A/B testing tools}
    \label{table:appendix:tools:review}
\end{table}

\subsection*{Literature and State of the Art (SOAT) technologies}
Our non-systematic literature review led us to state-of-the-art research in several key areas relevant to our solution approach (see table \ref{table:appendix:literature:review}). 
We identified research on UI prototyping, which can help us design and refine our solution's user interface to meet our users' needs. 
We also found research on low-code/no-code development, which can help us to build our solution quickly and efficiently. 
Additionally, we explored model-based software engineering, which can provide us with a structured approach to software development. 
We also discovered research on continuous experimentation, which can help us to test and improve our solution over time. 
Task-based usability testing was another key area of research we explored, which can help us to ensure that our solution is easy to use and meets the needs of our users. 
Finally, we looked into the LEAN development cycle, which can help us build our solution iteratively and incrementally while minimizing waste and maximizing user value. 
By incorporating the insights from these research areas into our solution approach, we hope to create a robust and effective solution that meets the needs of our users.
\label{section:appendix:literature}
\begin{table}[htbp!]
    \centering
    \begin{tabular}{| m{0.8em} || m{13em} | m{13em} | m{4em} | }
    \hline 
    \multicolumn{4}{|c|}{\textbf{Literature and \ac{soa} Research}} \\ 
    \hline
    \# & \textbf{Name}  &  \textbf{Type} &  \textbf{View} \\
    \hline
    1 & Rapid software prototyping &  UI Prototyping & \href{https://ieeexplore.ieee.org/abstract/document/183261}{Link} \\
    \hline 
    2 & Automating the software development process &  UI Prototyping & \href{https://ieeexplore.ieee.org/abstract/document/5387726}{Link}\\
    \hline 
    3 & User interface prototyping & UI Prototyping & \href{https://link.springer.com/content/pdf/10.1007/BFb0035808.pdf}{Link} \\
    \hline
    4 & Exploratory user interface design using scenarios and prototypes & UI Prototyping & \href{https://books.google.com/books?hl=de&lr=&id=BTxOtt4X920C&oi=fnd&pg=PA191&dq=Exploratory+user+interface+design+using+scenarios+and+prototypes&ots=OGrm6tv1Vn&sig=j0UzKvk5rZaGp8js6BLOvUdEl94}{Link} \\
    \hline
    5 & Low-code application platform & \ac{lcdp} & \href{https://aip.scitation.org/doi/abs/10.1063/5.0042213}{Link} \\
    \hline
    6 & Low Code vs No Code &  \ac{lcdp} / \ac{ncdp} & \href{https://blogs.bmc.com/low-code-vs-no-code/?print-posts=pdf}{Link} \\
    \hline 
    7 & No-code platform & \ac{ncdp} & \href{https://link.springer.com/article/10.1007/s12599-021-00726-8}{Link} \\
    \hline 
    8 & Adaptive model-driven user interface development systems & \ac{mbse} & \href{https://dl.acm.org/doi/abs/10.1145/2597999/}{Link}\\ 
    \hline 
    9 & OMG, Meta-Object facility & \ac{mbse} & \href{http://essay.utwente.nl/57286/}{Link} \\
    \hline
    10 & Turn User Goals into Task Scenarios for Usability Testing & Task Based Usability Testing & \href{https://www.nngroup.com/articles/task-scenarios-usability-testing/}{Link} \\
    \hline
    11 & But You Tested with Only 5 Users! & Task Based Usability Testing & \href{https://www.nngroup.com/articles/responding-skepticism-small-usability-tests/}{Link} \\
    \hline
    12 & Continuous experimentation and A/B testing: a mapping study & Split Testing & \href{https://dl.acm.org/doi/abs/10.1145/3194760.3194766/}{Link} \\
    \hline
    13 & Evolving UX: experimental product design with a CXO & Split Testing & \href{https://uxdesign.cc/evolving-ux-experimental-product-design-with-a-cxo-2c0865db80cc}{Link} \\
    \hline
    14 & Why the lean start-up changes everything? & LEAN & \href{https://hbr.org/2013/05/why-the-lean-start-up-changes-everything}{Link} \\
    \hline
    15 & Lean software development: A tutorial & LEAN & \href{https://ieeexplore.ieee.org/abstract/document/6226341}{Link} \\
    \hline
    16 & Continuous planning: an important aspect of agile and lean development & LEAN & \href{https://www.inderscienceonline.com/doi/abs/10.1504/IJASM.2015.070607}{Link} \\
    \hline
    \end{tabular}
    \caption[Literature review]{Table for different \ac{soa} research}
    \label{table:appendix:literature:review}
\end{table}