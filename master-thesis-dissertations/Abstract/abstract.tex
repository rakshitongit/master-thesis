% ************************** Thesis Abstract *****************************
% Use `abstract' as an option in the document class to print only the title page and the abstract.
\begin{abstract}
\ac{ui} is one of the essential aspects of software applications since it associates end-users with the applications' functionality. 
Therefore, the software is successful from the end user's perspective if it facilitates good interaction between users and the system. 
However, the traditional \ac{ui} prototyping techniques do not provide an easy exploration of alternative designs or efficient evaluation of user feedback. 
Hence, for obtaining user feedback, \ac{ui} experimentation can be an effective approach. 
\ac{ui} experimentation involves creating different UI variants and testing them with users to gather feedback on their usability and functionality. 
This process can help identify and address any issues or areas for improvement before the final product is released. 
By incorporating user feedback into the design process, software developers can increase the likelihood of creating a successful product that meets the needs and expectations of their target audience.
Currently, \ac{ui} prototyping and experimentation are developed in a non-systematic manner, leading to inefficient and inconsistent results.
Therefore, a model-based approach to \ac{ui} prototyping and experimentation is necessary based on a predefined model.

In this thesis, we conduct design science research to develop a UI prototyping tool tailored for \ac{ux} designers with the provision to conduct UI experiments.
This tool will give designers valuable insights into user preferences and behaviors, allowing them to refine and optimize their designs iteratively.
In the first step, we identify the problem of \ac{ui} prototyping and experimentation in the software development process. 
Second, we suggest conceptualizing a solution by developing Design Principles (DPs).
Third, we build a prototyping tool using the LEAN development cycle, a popular methodology in software development, allowing us to develop and iterate quickly. 
Fourth, we evaluate the prototype through reviews and usability testing using a user study with 15 participants.
Fifth, we gain design knowledge from the feedback of the participants of our user study, which suggest further iterations of the DPs.
Finally, we conclude that the developed tool and the proposed DPs can significantly improve the effectiveness and efficiency of \ac{ux} designers in creating successful software products. 
At the same time, the tool can also enable early user feedback, crucial in today's rapidly evolving market, impacting academic and practical applications.

\end{abstract}