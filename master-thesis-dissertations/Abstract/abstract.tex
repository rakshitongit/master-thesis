% ************************** Thesis Abstract *****************************
% Use `abstract' as an option in the document class to print only the title page and the abstract.
\begin{abstract}

\ac{ui} is one of the essential aspects of software applications since it associates end-users with the applications' functionality.
For interactive applications, the usability and convenience of the \ac{ui} are essential for achieving user acceptability.
Therefore, the software is successful from the end user's perspective if it facilitates good interaction between users and the system.
Hence, \ac{ui} prototyping is a crucial aspect of software development that helps create effective and user-friendly interfaces.
But the traditional \ac{ui} prototyping methods often rely on manual and time-consuming processes, which can result in inconsistencies and errors in the final product. 
At the same time, these methods do not allow for easy exploration of alternative designs or efficient evaluation of user feedback.
Therefore, a model-based approach to \ac{ui} prototyping and experimentation is necessary, in which \ac{ux} designers and software developers collaborate to create various \ac{ui} screens or variants based on a predefined model. 
These screens can be used to conduct split tests to evaluate their effectiveness, allowing for iterative improvements to the \ac{ui}.

This thesis proposes a model-based approach that leverages the \ac{mde} paradigm to create \ac{ui} prototypes and the \ac{doe} methodology to evaluate user feedback systematically.
This thesis follows a \ac{dsr} approach, including designing, implementing, and evaluating a custom-built prototype tool to support the proposed approach.
Our approach began by defining a set of requirements for our tool, which were concretized into a set of principles that guided the development of the tool and tracked its features. 
Then, to evaluate our proposed approach, we conducted a user study involving 15 participants who used the custom-built prototyping tool as a \ac{poc}.
And the results were analyzed through quantitative and qualitative methods.
The results indicate that the model-based approach improves the efficiency and effectiveness of the \ac{ui} prototyping process and helps capture and accurately evaluate user feedback. 
The study also highlights the limitations of the approach, including the need for a larger number of participants to enhance the validity of the results.
Finally, our results show that our model-based \ac{ui} prototyping experimentation approach can effectively enable collaboration between \ac{ux} designers and software developers, improving user satisfaction with the \ac{ui}.

\end{abstract}
